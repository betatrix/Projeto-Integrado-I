\label{resumo}
\chapter*{RESUMO}
A escolha do curso superior é um momento crucial na vida de muitos jovens, representando um marco significativo na trajetória educacional e profissional em suas vidas. Porém, o momento de transição da escola para a faculdade se torna um desafio para muitos estudantes da rede pública de ensino no Brasil, que frequentemente enfrentam barreiras socioeconômicas e não possuem o mesmo nível de acesso à informação quanto um estudante da rede particular, dificultando assim a tomada de decisões com relação ao seu futuro, e possivelmente gerando uma falta de perspectiva com relação ao acesso ao ensino superior de qualidade. Segundo pesquisa do \ac{inep}, em 2020, apenas 31,7\% dos jovens concluintes do ensino médio em escolas públicas ingressaram no ensino superior. Esse dado revela a necessidade de medidas com o objetivo de facilitar o acesso à informação e apoiar esses estudantes no momento da escolha de qual caminho seguir academicamente, com o intuito de abrir as portas das universidades públicas para essa parcela da população. Nesse contexto surge a Plataforma Vocco, que se propõe a ser uma ferramenta de apoio a esses jovens, fornecendo um guia a respeito de qual o curso mais indicado para o respectivo perfil de cada estudante, utilizando-se de um teste vocacional baseado na teoria do cientista John Holland, que mapeia as seis principais dimensões de interesses vocacionais, para designar o estudante à carreira mais adequada. Ademais, a Vocco se propõe a auxiliar o estudante com informações de quais faculdades públicas do país oferecem tais cursos, e como o estudante irá se manter no ensino superior, com informações de políticas de cotas e políticas de permanência (auxílios, moradia) das respectivas universidades. A metodologia adotada para o desenvolvimento da Vocco envolveu a análise da literatura existente para fundamentar as funcionalidades da plataforma e uma abordagem ágil para a criação e teste do protótipo, assegurando que a aplicação atenda às necessidades identificadas e seja eficaz na melhoria do acesso ao ensino superior. A principal base de dados para informações de cursos e \ac{ies} utilizada foi o \ac{e-mec}, já para os cursos e instituições técnicas foi utilizado o website oficial do Centro Paula Souza, como também do  \ac{if}.
Para obter as informações de nota do Mec, tanto das \ac{ies} quanto dos cursos superiores, foi utilizado também o \ac{e-mec} como principal fonte, já os dados da nota \ac{ideb} dos cursos técnicos teve como fonte o \ac{qedu}, administrado pelo Instituto Iede.
Além disso, a classificação de ambos os tipos de cursos, tanto superiores quanto técnicos foi definida pelo \ac{cnpq}, garantindo a precisão e a relevância das informações fornecidas.



\vspace{3\baselineskip}
\textit{\textbf{Palavras-chave:} curso superior; trajetória educacional; estudantes da rede pública; barreiras socioeconômicas}


\chapter*{ABSTRACT}
The choice of higher education is a crucial moment in the lives of many Choosing a higher education course is a crucial moment in the lives of many young people, representing a significant milestone in their educational and professional trajectories. However, the transition from school to college becomes a challenge for many students in Brazil’s public education system, who often face socioeconomic barriers and lack the same level of access to information as private school students. This disparity makes it difficult to make informed decisions about their future and can result in a lack of perspective on accessing quality higher education. According to research from \ac{inep}, in 2020, only 31.7\% of high school graduates from public schools enrolled in higher education. This data highlights the need for measures to facilitate access to information and support these students in choosing their academic path, aiming to open the doors of public universities to this segment of the population. In this context, the Vocco Platform emerges as a tool to support these young people by providing guidance on the most suitable course for each student's profile, using a vocational test based on John Holland’s theory, which maps the six main dimensions of vocational interests to match students with the most appropriate careers. Additionally, Vocco aims to assist students with information about which public colleges in the country offer these courses and how students can sustain themselves in higher education, including details on quota policies and support policies (scholarships, housing) at the respective universities. The methodology adopted for the development of Vocco involved analyzing existing literature to underpin the platform's features and using an agile approach to create and test the prototype, ensuring that the application meets identified needs and is effective in improving access to higher education. The primary database used for course and higher education institution information was \ac{e-mec}, while for technical courses and institutions, the official websites of Centro Paula Souza and \ac{if} were referenced. To obtain \ac{mec} ratings for both higher education institution and higher education courses, \ac{e-mec} was also used as the information source. For \ac{ideb} ratings of technical courses, the source was \ac{qedu}, managed by Instituto Iede. Additionally, the classification of both higher education and technical courses was defined by \ac{cnpq}, ensuring the accuracy and relevance of the information provided.

\vspace{3\baselineskip}
\textit{\textbf{Keywords:} higher education; educational trajectory; public school students; socioeconomic barriers}