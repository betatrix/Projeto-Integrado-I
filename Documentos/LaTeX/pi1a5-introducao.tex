\chapter{Introdução}
Ao longo dos anos, a tecnologia tem ampliado sua presença na vida de jovens e estudantes, participando também da sua vida educacional. Para Alves (\citeyear{alves2022tecnologia}), ela faz parte das grandes transformações que estão ocorrendo na educação, pois incorpora no ambiente educacional o acesso diversificado de informações e ferramentas digitais que podem ser utilizadas para gerar conhecimento. 

Essa realidade, abre caminho para a utilização de recursos tecnológicos como aliados no processo de escolha de carreira, apoiando na identificação e planejamento dos passos em direção ao futuro profissional, apresentando grande relevância para alunos de escola pública, inclusive no contexto de busca de cursos do ensino superior. Para esses alunos é importante ter conhecimento de opções que se apresentem mais viáveis e acessíveis em decorrência de sua condição socioeconômica, já que as famílias de menores faixas de renda têm na escola pública uma das poucas alternativas para a escolarização de seus filhos \cite{matos2012impacto}.

Contudo, em contraponto a esse cenário, mesmo que os estudantes de escola pública representem a maioria dos estudantes do ensino médio, os alunos da rede pública ainda são uma minoria no ensino superior \cite{alvarenga2012desafios}, sendo que os motivos variam entre as altas relações candidato/vaga das universidades públicas, falta de recursos financeiros para arcar com o ensino superior privado, dentre outros fatores como a dificuldade do acesso à informação à respeito de como entrar e se manter nas universidades públicas.

Além disso, apesar da quantidade de recursos digitais disponíveis, há uma carência de sites que consolidam informações vocacionais e de carreira, que atendam às especificidades dos alunos provenientes do sistema público de ensino. Essa falta de equilíbrio entre a orientação disponível e a realidade do indivíduo pode resultar em escolhas de carreira mal informadas e desconhecimento acerca de políticas afirmativas, fundamentais para assegurar o acesso ao ensino superior, perpetuando um ciclo de desinformação e  privando esses alunos do acesso ao ensino superior.

Nesse cenário, a Vocco surge como uma plataforma dedicada a atender as necessidades dos alunos de escolas públicas que buscam orientações sobre o ensino técnico e superior. A plataforma disponibilizará diversas informações sobre as \ac{ies}  públicas brasileiras e intituições de ensino técnico, incluindo formas de ingresso, meios de contato, endereços, políticas de cotas e programas de permanência estudantil. Esses auxílios são essenciais para que os alunos possam se manter durante o curso, ou graduação. Além disso, a Vocco oferecerá um teste vocacional para ajudar os alunos do ensino médio a escolherem a carreira mais adequada para seguir. A concepção e desenvolvimento da Vocco foram orientados por uma análise cuidadosa da literatura existente e por um processo iterativo, assegurando que a plataforma ofereça recursos relevantes e adequados às necessidades dos usuários.




\section{Justificativa}

A relevância deste sistema é evidenciada pela crescente demanda por serviços de orientação personalizados e acessíveis. Ao simplificar o acesso a informações pertinentes e oferecer orientação personalizada, o sistema não apenas preenche uma lacuna crítica nos recursos de auxílio à escolha profissional disponíveis, mas também contribui para a capacitação dos estudantes em tomar decisões mais alinhadas com suas habilidades e interesses.

\section{Objetivos}

Este projeto tem como objetivo desenvolver um sistema web que ofereça teste vocacional personalizado e gestão de informações sobre Universidades Públicas Federais do Brasil e Estaduais de São Paulo, como cursos oferecidos, formas de ingresso e políticas públicas de acesso e permanência. Pretendendo facilitar a conexão entre estudantes de escolas públicas ao acesso a Universidades Públicas, e o desenvolvimento de  um processo de escolha de ensino superior e de carreira mais preparado.



\section{Análise da Concorrência}

Embora haja uma variedade de plataformas de serviços de orientação vocacional disponíveis, quatro delas se destacam como potenciais concorrentes do projeto, sendo elas: Soutec, Kuau, Orientação Vocacional e Coaching e FutureMe.

\subsection{Aplicativo Soutec}
O objetivo principal do aplicativo Soutec,  é auxiliar os cidadãos, em especial os jovens, no planejamento de suas carreiras profissionais, por meio do suporte na identificação dos seus perfis, assim como na escolha do melhor curso técnico disponível para o desenvolvimento de suas competências. Para isso, o aplicativo Soutec disponibiliza testes especializados para identificação do perfil profissional, com detalhamento da profissão e recomendações de cursos técnicos baseados no \ac{cnct} também considerando a localização de interesse do usuário.
 Mesmo com as similaridades com a Plataforma Vocco, os focos são distintos. O público alvo que a Soutec busca é predominantemente as pessoas que desejam informações sobre cursos técnicos, fazendo com que o jovem que opte por seguir pelo ensino superior não encontre orientações e auxílio para seguir por tal caminho, já o foco da Vocco são jovens de baixa renda de escolas públicas que desejam tanto informações de cursos a nível superior quanto cursos técnicos. Além disso, a Vocco apresenta maiores vantagens para o usuário ao indicar não somente os cursos adequados para o perfil do jovem, como também qual instituição pública de ensino no Brasil oferece aquele curso, e quais são as políticas de permanência presentes em cada instituição para que o aluno consiga se manter no decorrer do curso.



\subsection{Kuau}
O Kuau é um aplicativo de orientação profissional que oferece uma metodologia inovadora, apelidada de "Netflix das profissões". Ele utiliza vídeos de curta duração com depoimentos de universitários, recém-formados e profissionais para explorar diversas carreiras. Seus diferenciais incluem um "termômetro de afinidade", que ajuda os alunos a definirem suas preferências enquanto assistem aos vídeos, e um certificado de proficiência, que avalia o aprendizado sobre cada profissão. O aplicativo também se integra ao Projeto de Vida das escolas, fornecendo relatórios e indicadores de acompanhamento.

O conceito da abordagem da Kuau é muito distante da abordagem usada pela Vocco, não oferecendo um teste vocacional  e focando somente na divulgação e detalhes das profissões que o usuário se interessa. Também não evidencia as informações sobre universidades públicas em que o usuário poderia ingressar para estudar tal profissão. Isso faz com que a Vocco se torne mais atrativa para os usuários que buscam um teste vocacional, já que nossa plataforma investe na centralização de informações de Instituições públicas de ensino no Brasil e suas políticas de permanência, além de uma análise da empregabilidade do curso indicado pela plataforma, fazendo com que os usuários não tenham somente a indicação de qual curso mais se adequa ao seu perfil como o “termômetro de afinidade” usado pela KUAU, mas também informa como o estudante conseguirá se manter financeiramente no curso, e quais as expectativas da empregabilidade no mercado.

\subsection{Orientação Vocacional e Coaching}
A plataforma oferece um método de orientação vocacional, projetado para ajudar os indivíduos a descobrir suas futuras profissões e definir seus percursos educacionais. Eles disponibilizam um \textit{ebook} gratuito e um teste vocacional online adaptado para os sistemas educacionais de Portugal, Brasil, Angola e Moçambique, baseado no modelo hexagonal de John Holland, que também será a base para a recomendação de cursos da Vocco. O site também apresenta recursos de orientação vocacional abrangentes, incluindo informações úteis para orientadores e orientadores.

A Orientação Vocacional e Coaching não oferece orientações para os estudantes em relação às universidades que disponibilizam os cursos indicados pela plataforma, focando somente na orientação vocacional com testes adaptados e a disponibilização de \textit{ebooks} gratuitos. Dessa forma a Vocco se destaca informando não somente um teste vocacional, mas sim informações da empregabilidade atual do curso indicado pelo teste, informações de qual instituição pública de ensino oferece tais cursos, e suas políticas de permanência, auxiliando o jovem na escolha de curso e em sua permanência.



\subsection{FutureMe}
A plataforma foca na orientação profissional autodirigida e gamificada para preparar estudantes para uma escolha profissional mais consciente. Eles atendem todo o Brasil e têm parcerias com mais de 150 escolas. A plataforma destaca a importância do autoconhecimento e oferece uma trilha gamificada que os alunos podem percorrer durante 7 a 10 aulas do Projeto de Vida, culminando em um evento de compartilhamento de experiências e entrega de certificados.

Por ser predominantemente uma plataforma para orientação profissional, a FutureMe também não auxilia o usuário com informações de possíveis universidades de ingresso, além disso  não disponibiliza um teste vocacional, mas sim uma orientação gamificada, sendo um foco diferente da plataforma Vocco. Essa plataforma se torna uma concorrente da Vocco por apresentar ao usuário um teste vocacional direcionando-o para uma potencial carreira adequada ao seu perfil, porém a Vocco se destaca ao informar não somente os cursos mais relevantes para o perfil do jovem, mas também qual a expectativa de empregabilidade daquele curso, quais instituições públicas do Brasil oferecem-no e quais a políticas públicas estão presentes naquela determinada instituição, auxiliando o jovem a encontrar uma potencial carreira e meios de se manter financeiramente no decorrer do curso.







% % \section{Papéis e responsabilidades}

% Na disciplina existem diversos papéis que são assumidos pelos participantes, o entendimento desses papéis permite atingir corretamente os objetivos da disciplina:
% \begin{itemize}
%     \item \textbf{Estudante} - Deve desenvolver as atividades da disciplina seguindo os preceitos deste documento e orientações passadas pelos professores, colaborando para o sucesso do projeto desenvolvido pela equipe;


%     \item \textbf{Equipe} - Segundo \cite{EQUIPES}: 
%     \begin{citacao}“Um grupo de pessoas com alto grau de interdependência está direcionado para a realização de uma meta ou para a conclusão de uma tarefa, cria-se o conceito de EQUIPE. Em outra palavras, membros de uma equipe concordam com uma meta e concordam que a única maneira de alcançar essa meta é trabalhar em conjunto". 
%     \end{citacao}
%     Desta forma, as equipes são compostas por um número definido de estudantes, que tem como objetivo a concretização do trabalho da disciplina.
    
%     Algumas outras definições de equipes e vídeos de apoio podem ser encontrados em: \dicasIvan{equipes}
    

%     \item \textbf{Professor} - Tem o papel de orientar e avaliar, buscando atingir os objetivos da disciplina;
    
%     \justificativa{Eles precisam buscar informações antes de nos questionar e vamos atender de acordo com requisições, não devemos influenciar nas decisões se não formos questionados ou consultados}
    

%     \item \textbf{Cliente} - Os professores assumem o papel de cliente do projeto e devem ser consultados como um cliente real. Ao desenvolver uma aplicação para resolver um problema real e tendo acesso a usuários reais o projeto pode evoluir muito pois passa por diversas visões em relação ao problema. Uma equipe também pode assumir o papel de cliente de outra equipe desde que não entrem em conflito com as decisões dos clientes principais que são os Professores;

%     \item \textbf{Banca Examinadora} - O trabalho é apresentado para uma banca que vai avaliar tanto os documentos demonstrando o desenvolvimento como a aplicação em execução. Essa banca é composta pelos professores da disciplina e convidado(s).
    
% \end{itemize}


% \section{Avaliações}

Todos as atividades desenvolvidas durante o projeto são avaliadas pelos professores, prazos de entregas são considerados como datas de Provas / Avaliações tradicionais e portanto devem ser respeitados.

Além da documentação e da execução da aplicação, são avaliados os modelos estáticos e dinâmicos da aplicação, bem com a relação entre modelos, aplicação e objetivos do projeto.

A divisão das atividades desempenhadas por cada elemento da equipe deve ser documentada no trabalho. A avaliação pode ser individualizada, conforme as atividades desempenhadas por cada aluno ao longo do projeto.

Cada equipe também faz avaliações de seus membros, cada participante avalia todos os membros de sua equipe (incluindo ele mesmo). Essas avaliações também poderão ser consideradas pelos professores ao definir a nota individual de cada participante.




% \section{Comunicação entre alunos e professores}
Existem diversos canais de comunicação que poderão ser utilizados durante o desenvolvimento da disciplina, conforme a seguir descrito.
\begin{itemize}
    \item Comunicador do \ac{suap} onde os professores muitas vezes enviam comunicados oficiais que precisam de registro;
    
    \item Curso definido no Moodle da disciplina onde algumas atividades devem ser entregues e também permite o envio de mensagens;
    
    \item E-mail dos alunos para os professores com dúvidas especificas (sempre enviar com cópia para ambos professores e indicar claramente qual a turma/equipe que faz parte, pois os professores tem projetos de diversas turmas e nem sempre os e-mails chegam com o nome correto do aluno);
    
    \item Grupo no Telegram que permite a comunicação entre todos participantes da turma, dúvidas genéricas devem ser feitas principalmente por esse canal pois permitem que todos tenham acesso a informação. Nesse grupo são enviados os links para as aulas/conversas síncronas da disciplina;
    
    \item Ferramentas de conferencia (Meet, RNP, Teams, Telegram  etc) para conversas síncronas.
    
\end{itemize}

É importante lembrar que algumas ferramentas devem ser utilizadas com cuidado, não se deve enviar mensagem com notificação de madrugada por exemplo. No Telegram é possível agendar o envio de uma mensagem ou até enviar sem a notificação, bastando escolher isso no momento do envio.

Existe um canal genérico de projetos (IFSP-SPO-Projetos) onde algumas informações gerais que atendem alunos do ensino médio e superior são publicadas \url{https://t.me/joinchat/AAAAAET-oEt6v2nyhgx2CQ}.





\justificativa{O Telegram não dá acesso ao número de telefone se o usuário não desejar, assim respeita um pouco o que a própria escola faz onde não temos acesso aos números, os grupos tem o histórico disponível e assim os alunos tem acesso ao que já aconteceu}




