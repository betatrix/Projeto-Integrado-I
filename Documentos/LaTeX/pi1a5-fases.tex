\chapter{REVISÃO BIBLIOGRÁFICA}
\label{fases-da-disciplina}


\section{O processo de escolha profissional para jovens do Ensino Médio}

A transição do ensino médio para a vida adulta é uma fase repleta de desafios, e a escolha da carreira profissional é uma das decisões mais impactantes que os jovens enfrentam durante esse período, uma escolha que muitas vezes não é estudada com a profundidade necessária. O artigo conduzido por Talisson de Sousa Lopes e Sônia Christo Aleixo A. Brito   oferece análises valiosas sobre os fatores que influenciam essa escolha. A pesquisa, realizada com 78 alunos do terceiro ano do ensino médio, busca compreender os motivadores e influenciadores por trás das decisões de carreira desses jovens.

Inicialmente, o estudo revela que a situação financeira é um dos principais impulsionadores para os jovens buscarem uma colocação no mercado de trabalho ou ingressarem em cursos profissionalizantes ou superiores. Essa busca por melhores oportunidades é frequentemente encorajada pelos pais e familiares dos estudantes.
Além disso, a pesquisa revela que a maioria dos participantes está na faixa etária de 16 a 19 anos, uma fase em que a pressão para definir o futuro profissional é intensa. Um aspecto interessante levantado pelo estudo é que muitos alunos já têm uma ideia clara sobre seu futuro profissional após a conclusão do ensino médio, sendo que grande parte deles expressa o desejo de cursar uma faculdade para alcançar suas aspirações profissionais \cite{lopes2022fim}.

No entanto, a pesquisa também destaca os desafios enfrentados pelos jovens nesse processo. A necessidade financeira surge como um fator limitante \cite{lopes2022fim}, levando alguns estudantes a aceitarem empregos que não correspondem às suas aspirações, comprometendo suas oportunidades de crescimento profissional. Além disso, as condições socioeconômicas das famílias dos alunos impactam diretamente seu desempenho acadêmico, conforme evidenciado por estudos anteriores citados no artigo.

A influência familiar na escolha da profissão também é ressaltada \cite{lopes2022fim}, com muitos alunos seguindo carreiras que se alinham aos interesses de seus pais. Isso evidencia a dificuldade dos jovens em definir uma carreira própria, especialmente quando enfrentam barreiras de acesso à informação durante essa fase de transição.


\section{Testes Vocacionais como Ferramenta de Orientação Profissional}

Os testes vocacionais têm sido amplamente utilizados como uma ferramenta de orientação profissional, auxiliando os jovens na identificação de suas habilidades, interesses e aptidões. Pesquisas evidenciam que a aplicação de testes vocacionais pode contribuir significativamente para a tomada de decisão dos estudantes em relação à escolha da carreira e do curso universitário. Além disso, esses testes podem ajudar a reduzir a evasão universitária, direcionando os alunos para áreas de maior afinidade e interesse.

O modelo tipológico de personalidades vocacionais proposto por Holland (1997) tem sido fundamental na pesquisa e prática de avaliação de interesses ao longo das últimas décadas \cite{de2006relaccao}. Holland postula que os interesses vocacionais são uma expressão da personalidade, e que pessoas dedicadas a uma mesma ocupação tendem a possuir personalidades similares. Seu modelo identifica seis tipos de personalidades: realista (R), investigativo (I), artístico (A), social (S), empreendedor (E) e convencional (C). Esses tipos de personalidades criam ambientes físicos e interpessoais distintos, que podem ser categorizados pela tipologia RIASEC. Cada tipo de ambiente tem estratégias preferidas de solução de problemas e estilos de interação interpessoal específicos (Holland, 1997). 

Magalhães, Martinuzzi, Teixeira (2004) e Martins (1978) forneceram descrições recentes dos seis tipos de personalidades vocacionais, enquanto \cite{de2006relaccao} ofereceu uma revisão abrangente do estado atual das pesquisas sobre o modelo. Essas pesquisas destacam a importância do modelo de Holland na compreensão das relações entre personalidade, trabalho e desenvolvimento de carreira. Os resultados sugerem que os diferentes tipos vocacionais estão de fato associados a padrões distintos de comportamento interpessoal. Por exemplo, os tipos Artísticos tendem a exibir um estilo mais voltado para a originalidade e a expressão pessoal, enquanto os tipos Realistas são mais reservados e avessos à interação social. Essas descobertas são consistentes com as descrições feitas por Holland sobre as características de cada tipo.

Além disso, o estudo destaca a importância de considerar os estilos interpessoais ao orientar indivíduos em suas escolhas de carreira. Em vez de simplesmente vincular os interesses vocacionais a ocupações específicas, deve-se procurar entender como nossos interesses refletem metas e valores pessoais e como essas características são expressas no ambiente de trabalho.  No entanto, o estudo reconhece que as tipologias de personalidade têm suas limitações e não devem ser vistas como rígidas ou definitivas. Em vez disso, elas fornecem uma estrutura útil para entender as semelhanças e diferenças entre as pessoas, enquanto ainda reconhecem a complexidade e a individualidade de cada caso.


\section{Desigualdade no acesso ao ensino superior}

O acesso à educação é um tema fundamental em qualquer sociedade, refletindo diretamente no desenvolvimento humano e socioeconômico de um país. No Brasil, país marcado por profundas desigualdades sociais e econômicas, a análise desse acesso, especialmente no que tange ao ensino superior, revela tanto avanços quanto desafios persistentes. Nos últimos anos, o debate sobre o acesso à educação no Brasil tem ganhado relevância tanto nas esferas acadêmicas quanto nas políticas públicas. Neste contexto, o estudo conduzido por Cibele Yahn oferece uma análise sobre essa questão, destacando os principais aspectos que influenciam o acesso à educação e apontando para a necessidade de políticas públicas eficazes para promover uma verdadeira democratização do ensino.

 \cite{de2012acesso} começa por contextualizar o cenário educacional brasileiro, ressaltando a transformação significativa que ocorreu a partir dos anos 90, com a universalização do ensino fundamental e o aumento substancial das matrículas no ensino médio e superior. Entretanto, apesar desses avanços, a autora destaca que o acesso ao ensino superior ainda é limitado, especialmente quando comparado a países mais desenvolvidos.

Um dos aspectos mais importantes da análise é a diferenciação entre os jovens que acessam o ensino superior e aqueles que não conseguem, com base em variáveis como renda familiar e raça/cor autodeclarada. Os resultados obtidos evidenciam as desigualdades socioeconômicas e raciais persistentes no sistema educacional brasileiro, mostrando que os jovens mais pobres e os autodeclarados não brancos enfrentam maiores dificuldades de acesso  \cite{de2012acesso}.

A autora também aborda o atraso escolar como uma barreira significativa para o acesso ao ensino superior, destacando que uma parcela considerável dos jovens brasileiros não possui os requisitos educacionais formais para ingressar nesse nível de ensino, ou seja, possuem o ensino fundamental e/ou médio incompleto. Além disso, a análise da evolução do acesso à educação ao longo dos anos revela tanto avanços quanto desafios, ressaltando a necessidade de ampliar o acesso e melhorar a qualidade da educação básica.

Outro aspecto analisado foi a relação entre renda familiar, raça/cor autodeclarada e acesso à educação. Os dados apresentados pela autora demonstram claramente que as desigualdades no acesso ao ensino superior são fortemente influenciadas pela renda familiar, mas também têm uma dimensão racial importante, com os jovens não brancos enfrentando maiores obstáculos de acesso.

Por fim, \cite{de2012acesso} destaca a importância de políticas públicas voltadas para a promoção do acesso e da qualidade da educação, especialmente no ensino médio, como uma estratégia fundamental para aumentar o acesso ao ensino superior. A análise dos resultados do Enem é apresentada como um indicador crucial do nível de aprendizado dos jovens e da demanda qualificada para o ensino superior, destacando a necessidade de investimentos contínuos na melhoria do ensino básico para promover uma verdadeira democratização do acesso à educação no Brasil.
