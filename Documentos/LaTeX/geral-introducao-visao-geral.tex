\section{Análise da Concorrência}

Embora haja uma variedade de plataformas de serviços de orientação vocacional disponíveis, quatro delas se destacam como potenciais concorrentes do projeto, sendo elas: Soutec, Kuau, Orientação Vocacional e Coaching e FutureMe.

\subsection{Aplicativo Soutec}
O objetivo principal do aplicativo Soutec,  é auxiliar os cidadãos, em especial os jovens, no planejamento de suas carreiras profissionais, por meio do suporte na identificação dos seus perfis, assim como na escolha do melhor curso técnico disponível para o desenvolvimento de suas competências. Para isso, o aplicativo Soutec disponibiliza testes especializados para identificação do perfil profissional, com detalhamento da profissão e recomendações de cursos técnicos baseados no Catálogo Nacional de Cursos Técnicos (CNCT) e considerando a localização de interesse do usuário.
 Mesmo com as similaridades com a Plataforma Vocco, o foco da Soutec é predominantemente para o público que deseja informações sobre cursos técnicos, fazendo com que o jovem que opte por seguir pelo ensino superior não encontre orientações e auxílio para seguir por tal caminho.


\subsection{Kuau}
O Kuau é um aplicativo de orientação profissional que oferece uma metodologia inovadora, apelidada de "Netflix das profissões". Ele utiliza vídeos de curta duração com depoimentos de universitários, recém-formados e profissionais para explorar diversas carreiras. Seus diferenciais incluem um "termômetro de afinidade", que ajuda os alunos a definirem suas preferências enquanto assistem aos vídeos, e um certificado de proficiência, que avalia o aprendizado sobre cada profissão. O aplicativo também se integra ao Projeto de Vida das escolas, fornecendo relatórios e indicadores de acompanhamento​​.

O conceito da abordagem da Kuau é muito distante da abordagem usada pela Vocco, não oferecendo um teste vocacional  e focando somente na divulgação e detalhes das profissões que o usuário se interessa. Também não evidencia as informações sobre universidades públicas em que o usuário poderia ingressar para estudar tal profissão.

\subsection{Orientação Vocacional e Coaching}
A plataforma oferece um método de orientação vocacional, projetado para ajudar os indivíduos a descobrir suas futuras profissões e definir seus percursos educacionais. Eles disponibilizam um ebook gratuito e um teste vocacional online adaptado para os sistemas educacionais de Portugal, Brasil, Angola e Moçambique, baseado no modelo hexagonal de John Holland. O site também apresenta recursos de orientação vocacional abrangentes, incluindo informações úteis para orientadores e orientadores​​.

A Orientação Vocacional e Coaching não oferece orientações para os estudantes em relação às universidades que disponibilizam os cursos indicados pela plataforma, focando somente na orientação vocacional com testes adaptados e a disponibilização de ebooks gratuitos.


\subsection{FutureMe}
A plataforma foca na orientação profissional autodirigida e gamificada para preparar estudantes para uma escolha profissional mais consciente. Eles atendem todo o Brasil e têm parcerias com mais de 150 escolas. A plataforma destaca a importância do autoconhecimento e oferece uma trilha gamificada que os alunos podem percorrer durante 7 a 10 aulas do Projeto de Vida, culminando em um evento de compartilhamento de experiências e entrega de certificados​​.

Por ser predominantemente uma plataforma para orientação profissional, a FutureMe também não auxilia o usuário com informações de possíveis universidades de ingresso, além disso  não disponibiliza um teste vocacional, mas sim uma orientação gamificada, sendo um foco diferente da plataforma Vocco 




