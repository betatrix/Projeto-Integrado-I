\section{Análise da Concorrência}

Embora haja uma variedade de plataformas de serviços de orientação vocacional disponíveis, quatro delas se destacam como potenciais concorrentes do projeto, sendo elas: Soutec, Kuau, Orientação Vocacional e Coaching e FutureMe.

\subsection{Aplicativo Soutec}
O objetivo principal do aplicativo Soutec,  é auxiliar os cidadãos, em especial os jovens, no planejamento de suas carreiras profissionais, por meio do suporte na identificação dos seus perfis, assim como na escolha do melhor curso técnico disponível para o desenvolvimento de suas competências. Para isso, o aplicativo Soutec disponibiliza testes especializados para identificação do perfil profissional, com detalhamento da profissão e recomendações de cursos técnicos baseados no \ac{cnct} também considerando a localização de interesse do usuário.
 Mesmo com as similaridades com a Plataforma Vocco, os focos são distintos. O público alvo que a Soutec busca é predominantemente as pessoas que desejam informações sobre cursos técnicos, fazendo com que o jovem que opte por seguir pelo ensino superior não encontre orientações e auxílio para seguir por tal caminho, já o foco da Vocco são jovens de baixa renda de escolas públicas que desejam tanto informações de cursos a nível superior quanto cursos técnicos. Além disso, a Vocco apresenta maiores vantagens para o usuário ao indicar não somente os cursos adequados para o perfil do jovem, como também qual instituição pública de ensino no Brasil oferece aquele curso, e quais são as políticas de permanência presentes em cada instituição para que o aluno consiga se manter no decorrer do curso.



\subsection{Kuau}
O Kuau é um aplicativo de orientação profissional que oferece uma metodologia inovadora, apelidada de "Netflix das profissões". Ele utiliza vídeos de curta duração com depoimentos de universitários, recém-formados e profissionais para explorar diversas carreiras. Seus diferenciais incluem um "termômetro de afinidade", que ajuda os alunos a definirem suas preferências enquanto assistem aos vídeos, e um certificado de proficiência, que avalia o aprendizado sobre cada profissão. O aplicativo também se integra ao Projeto de Vida das escolas, fornecendo relatórios e indicadores de acompanhamento.

O conceito da abordagem da Kuau é muito distante da abordagem usada pela Vocco, não oferecendo um teste vocacional  e focando somente na divulgação e detalhes das profissões que o usuário se interessa. Também não evidencia as informações sobre universidades públicas em que o usuário poderia ingressar para estudar tal profissão. Isso faz com que a Vocco se torne mais atrativa para os usuários que buscam um teste vocacional, já que nossa plataforma investe na centralização de informações de Instituições públicas de ensino no Brasil e suas políticas de permanência, além de uma análise da empregabilidade do curso indicado pela plataforma, fazendo com que os usuários não tenham somente a indicação de qual curso mais se adequa ao seu perfil como o “termômetro de afinidade” usado pela KUAU, mas também informa como o estudante conseguirá se manter financeiramente no curso, e quais as expectativas da empregabilidade no mercado.

\subsection{Orientação Vocacional e Coaching}
A plataforma oferece um método de orientação vocacional, projetado para ajudar os indivíduos a descobrir suas futuras profissões e definir seus percursos educacionais. Eles disponibilizam um \textit{ebook} gratuito e um teste vocacional online adaptado para os sistemas educacionais de Portugal, Brasil, Angola e Moçambique, baseado no modelo hexagonal de John Holland, que também será a base para a recomendação de cursos da Vocco. O site também apresenta recursos de orientação vocacional abrangentes, incluindo informações úteis para orientadores e orientadores.

A Orientação Vocacional e Coaching não oferece orientações para os estudantes em relação às universidades que disponibilizam os cursos indicados pela plataforma, focando somente na orientação vocacional com testes adaptados e a disponibilização de \textit{ebooks} gratuitos. Dessa forma a Vocco se destaca informando não somente um teste vocacional, mas sim informações da empregabilidade atual do curso indicado pelo teste, informações de qual instituição pública de ensino oferece tais cursos, e suas políticas de permanência, auxiliando o jovem na escolha de curso e em sua permanência.



\subsection{FutureMe}
A plataforma foca na orientação profissional autodirigida e gamificada para preparar estudantes para uma escolha profissional mais consciente. Eles atendem todo o Brasil e têm parcerias com mais de 150 escolas. A plataforma destaca a importância do autoconhecimento e oferece uma trilha gamificada que os alunos podem percorrer durante 7 a 10 aulas do Projeto de Vida, culminando em um evento de compartilhamento de experiências e entrega de certificados.

Por ser predominantemente uma plataforma para orientação profissional, a FutureMe também não auxilia o usuário com informações de possíveis universidades de ingresso, além disso  não disponibiliza um teste vocacional, mas sim uma orientação gamificada, sendo um foco diferente da plataforma Vocco. Essa plataforma se torna uma concorrente da Vocco por apresentar ao usuário um teste vocacional direcionando-o para uma potencial carreira adequada ao seu perfil, porém a Vocco se destaca ao informar não somente os cursos mais relevantes para o perfil do jovem, mas também qual a expectativa de empregabilidade daquele curso, quais instituições públicas do Brasil oferecem-no e quais a políticas públicas estão presentes naquela determinada instituição, auxiliando o jovem a encontrar uma potencial carreira e meios de se manter financeiramente no decorrer do curso.






