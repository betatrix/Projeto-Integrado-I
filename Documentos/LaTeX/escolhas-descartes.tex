\chapter{DECISÕES}
\label{escolhas-descartes}
A seguir, serão justificadas as escolhas feitas para a elaboração do projeto, incluindo as tecnologias utilizadas e as funcionalidades que optou-se por não implementar ou que serão implementadas em versões futuras da aplicação.

\section{Escolhas}

Nesta seção, serão detalhadas as principais decisões técnicas tomadas durante a elaboração do projeto. Serão abordadas as razões por trás da seleção das tecnologias utilizadas, destacando os benefícios e vantagens que cada uma oferece para o desenvolvimento, manutenção e escalabilidade da aplicação.

\subsection{Utilização do Java}

A escolha de utilizar Java como a linguagem de programação para o \textit{back-end} foi baseada em sua robustez, segurança e ampla adoção no mercado. Java oferece uma vasta gama de bibliotecas e \textit{frameworks} que facilitam o desenvolvimento de aplicações escaláveis e de alto desempenho. Além disso, sua portabilidade e forte suporte comunitário garantem uma base sólida para a manutenção e evolução contínua do projeto.

\subsection{Utilização do Spring Boot}

O Spring Boot foi escolhido por sua capacidade de simplificar o desenvolvimento de aplicações Java baseadas em Spring. Ele oferece uma configuração mínima, permitindo que os desenvolvedores se concentrem nas funcionalidades da aplicação em vez de se preocupar com a infraestrutura. Além disso, o Spring Boot fornece uma série de ferramentas integradas para monitoramento, segurança e \textit{deploy}, tornando o processo de desenvolvimento mais eficiente e ágil.

\subsection{Utilização do TypeScript}

A adoção do TypeScript foi motivada pela necessidade de aumentar a produtividade e a qualidade do código no desenvolvimento \textit{front-end}. TypeScript, ao adicionar tipagem estática ao JavaScript, ajuda a prevenir erros comuns durante a compilação, melhorando a robustez do código. Além disso, sua integração com ferramentas de desenvolvimento modernas e a capacidade de utilizar recursos mais avançados do JavaScript facilitam a criação de aplicações web complexas e escaláveis.

\subsection{Utilização do React}

O React foi escolhido como a biblioteca principal para a construção da interface do usuário devido à sua eficiência e flexibilidade. Com sua abordagem baseada em componentes, o React permite a criação de interfaces de usuário reutilizáveis e altamente interativas. Além disso, sua forte comunidade e vasta quantidade de recursos e bibliotecas auxiliares tornam o desenvolvimento mais rápido e eficaz, garantindo uma experiência de usuário final mais rica e responsiva.

\section{Descartes}

Durante a fase de elaboração do projeto, decidiu-se descartar a funcionalidade de recomendar cursos e instituições estrangeiras que aceitam a nota do \ac{enem}. A principal razão para essa decisão foi a complexidade adicional envolvida na integração e atualização constante das informações de instituições estrangeiras. Além disso, a maioria dos usuários da plataforma está interessada em oportunidades de educação dentro do Brasil, tornando essa funcionalidade de menor prioridade. Acredita-se que focar exclusivamente em cursos e universidades brasileiras permite uma experiência de usuário mais direcionada, alinhada com os principais objetivos do projeto.


\section{Implementações futuras}

Para o desenvolvimento contínuo da plataforma e futuras entregas, estão previstas várias melhorias e novas funcionalidades que irão aprimorar a experiência do usuário e expandir a extensão dos dados disponíveis. Entre essas futuras implementações, destacam-se:

\subsection{Integração com API do e-MEC}
Uma funcionalidade planejada para otimizar a atualização dos dados das instituições é a integração com uma \ac{api} que permitiria buscar informações diretamente no site do e-MEC. Esta implementação estava prevista para ser desenvolvida no segundo semestre, com o objetivo de automatizar a atualização dos dados de universidades e cursos, garantindo que as informações fossem sempre atuais e precisas. No entanto, devido a complicações técnicas durante a integração e à limitação de tempo, não foi possível concluir essa implementação conforme o cronograma estabelecido. Caso seja viabilizada futuramente, a \ac{api} permitirá a obtenção de dados detalhados sobre as instituições de ensino superior, seus cursos, localizações, avaliações, entre outros. A atualização dos dados seguiria a regra de negócio proposta de atualização semestral, sincronizada com as datas de abertura dos processos seletivos das universidades, como o \ac{sisu} e o \ac{prouni}, garantindo que as informações disponibilizadas aos usuários estivessem sempre atualizadas e alinhadas aos períodos de ingresso nas instituições de ensino.

\subsection{Criação de Fórum para Discussão sobre Instituições}
Uma funcionalidade futura planejada é a criação de um fórum dentro da plataforma, permitindo que os próprios alunos possam discutir sobre as instituições de ensino cadastradas. Essa ferramenta possibilitará que os estudantes compartilhem experiências, opiniões e acrescentem informações sobre as universidades e cursos, ajudando a enriquecer o conteúdo da plataforma. A ideia é que os usuários possam reitificar dados já presentes e contribuir com informações novas, tornando a base de dados ainda mais rica e precisa. Com isso, espera-se criar uma comunidade colaborativa onde estudantes possam trocar informações valiosas, além de melhorar a transparência e confiabilidade dos dados apresentados sobre as instituições.

\subsection{Desenvolvimento de Filtros de Recomendação de Universidades}
Outra funcionalidade que está sendo planejada é o desenvolvimento de filtros de recomendação de universidades baseados na localização do estudante. Ao fornecer sua localização ou preferências geográficas, o usuário poderá receber sugestões personalizadas de instituições de ensino próximas a ele, facilitando o processo de escolha de uma universidade. Este filtro consideraria também outros fatores, como a proximidade com centros urbanos, custos de deslocamento e opções de moradia nas imediações, proporcionando ao estudante uma experiência de pesquisa mais prática e alinhada com suas necessidades regionais.

\subsection{Opção de Favoritar Cursos e Instituições}
Outra melhoria prevista para o futuro é a possibilidade de o estudante poder favoritar seus cursos e instituições de ensino preferidos. Com essa funcionalidade, os usuários poderão marcar suas opções mais interessantes para acessá-las facilmente em momentos posteriores, sem a necessidade de refazer buscas. A criação de uma área de "favoritos" permitirá que os estudantes acompanhem as informações mais relevantes para sua decisão, como detalhes dos cursos, políticas públicas, e forma de ingresso. Essa funcionalidade visa otimizar a navegação na plataforma, tornando o processo de seleção mais rápido e personalizado para cada estudante.

\subsection{Inclusão de Dados de Instituições Técnicas e seus Cursos}
Atualmente, a plataforma concentra-se nas informações sobre as ETECs e seus cursos. No entanto, em futuras implementações, espera-se expandir o banco de dados para incluir instituições técnicas de todo o Brasil e os cursos que elas oferecem. Essa ampliação permitirá que a plataforma atenda a um público mais diversificado, abrangendo diferentes tipos de formação técnica e profissional em diversas regiões do país, proporcionando uma maior variedade de opções para os usuários interessados nesse nível de educação.

