\chapter{GESTÃO DO PROJETO}
\label{requisitos-aplicacoes}

\section{Metodologia de gestão de projeto}

A metodologia de gestão de projeto definida pela equipe foi a metodologia \textbf{Scrum}, pois proporciona um framework flexível e adaptativo que se alinha bem com as necessidades dinâmicas do desenvolvimento de software. O Scrum permite uma abordagem iterativa e incremental, o que possibilita a entrega rápida e frequente de funcionalidades, garantindo uma maior capacidade de resposta às mudanças nos requisitos do cliente e do mercado. Além disso, o Scrum promove uma cultura de transparência, colaboração e auto-organização dentro da equipe, incentivando a responsabilidade compartilhada e o foco na entrega de valor para o cliente. Outros benefícios incluem a redução de riscos, a melhoria da comunicação e a maximização da satisfação do cliente ao longo do ciclo de vida do projeto. Em resumo, a adoção do Scrum como metodologia de projeto oferece uma série de vantagens que podem contribuir significativamente para o sucesso e a qualidade do desenvolvimento de software.

 É importante ressaltar que o Scrum é uma das muitas abordagens dentro do conjunto de \textbf{metodologias ágeis}, que surgiram como uma resposta à rigidez e à inadequação das metodologias tradicionais de desenvolvimento de software. As metodologias ágeis valorizam indivíduos e interações mais do que processos e ferramentas, adaptando-se a mudanças mais do que seguindo um plano rígido. No contexto do Scrum, essa agilidade é alcançada por meio de Sprints, reuniões diárias, retrospectivas e um foco constante na entrega incremental de valor ao cliente. 

\section{Papéis e Atribuições}

\begin{itemize}
    \item \textbf{Scrum Master:}
O Scrum Master, como líder servo da equipe, desempenha um papel crucial na remoção de obstáculos que possam impedir o progresso do projeto. Sua responsabilidade inclui facilitar reuniões, garantir a adesão aos princípios e práticas do Scrum, e promover um ambiente de trabalho colaborativo e auto-organizável. Além disso, o Scrum Master atua como um mentor, capacitando a equipe a tomar decisões e resolver problemas de forma independente. 
    \begin{itemize}
        \item Dentro do nosso projeto, esse papel foi desempenhado pela integrante Tamiris Delfino.
    \end{itemize}
    
    \item \textbf{Product Owner:}
O Product Owner é o representante dos stakeholders e tem a responsabilidade de maximizar o valor do produto. Isso envolve a definição e priorização do backlog do produto, garantindo que os requisitos do cliente sejam compreendidos e atendidos pela equipe de desenvolvimento. O Product Owner deve ter uma visão clara dos objetivos do negócio e trabalhar em estreita colaboração com a equipe para garantir que o produto seja desenvolvido de acordo com esses objetivos.
\begin{itemize}
        \item Dentro do nosso projeto, esse papel foi desempenhado pela integrante Letícia Baião.
    \end{itemize}

    \item \textbf{Desenvolvedores:} 
Os Desenvolvedores são os membros da equipe responsáveis por transformar os itens do backlog do produto em incrementos potencialmente entregáveis do produto. Eles são auto-organizáveis e colaboram de perto com o Product Owner para entender os requisitos do produto. Os Desenvolvedores têm a responsabilidade de garantir que o trabalho seja concluído dentro do prazo estabelecido para cada Sprint, bem como de buscar continuamente maneiras de melhorar o processo de desenvolvimento.
    \begin{itemize}
        \item Dentro do nosso projeto, esse papel foi desempenhado por todas as integrantes, dada a exigência de todas se envolverem com o desenvolvimento e o prazo para a entrega do projeto, que torna enviável a codificação ser feita por apenas três integrantes.
    \end{itemize}
\end{itemize}

\subsection{Ferramentas de gestão utilizadas}

No projeto, optamos por utilizar o \textbf{Jira} como nossa ferramenta de gerenciamento de projeto. O Jira foi escolhido devido à sua ampla gama de funcionalidades que atendem às necessidades específicas de gestão de projetos ágeis. Algumas das características chave do Jira incluem sua capacidade de criar e acompanhar histórias de usuário, gerenciar e priorizar o backlog do produto, atribuir tarefas aos membros da equipe, monitorar o progresso do projeto através de quadros Kanban e Scrum, e gerar relatórios sobre o desempenho da equipe e do projeto.

\subsection{Canais de comunicação}

No projeto, optamos por utilizar o WhatsApp como meio de comunicação para coordenar e agendar nossas reuniões. Já para a realização das reuniões semanais, escolhemos a plataforma Meet. Através dela, conseguimos conduzir nossas reuniões de forma remota, possibilitando a participação de todos os membros da equipe, mesmo em localidades diferentes.

Para facilitar o compartilhamento e acompanhamento do histórico de desenvolvimento do projeto, foi criado o blog ADA'S TECH IFSP. Este blog serve como um repositório de informações sobre a evolução do projeto, permitindo o acompanhamento das etapas. O acesso ao Blog está disponível no QR Code abaixo:

\begin{figure}[ht]
        \centering
        \includegraphics[scale=0.5]{qrCode-blog.png}
        \caption{Código QR Blog}
        \label{fig:enter-label}
    \end{figure}


Além disso, foi criado o Canal no YouTube @AdasTechIFSP, destinado à publicação de todos os vídeos produzidos ao longo do desenvolvimento do projeto, incluindo apresentações e vídeos do Gource. O acesso ao Canal está disponível no QR Code abaixo:

\begin{figure}[ht]
        \centering
        \includegraphics[scale=0.5]{qrCode-yotube.png}
        \caption{Código QR Canal YouTube}
        \label{fig:enter-label}
    \end{figure}