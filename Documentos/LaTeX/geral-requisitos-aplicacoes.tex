\chapter{GESTÃO DO PROJETO}
\label{requisitos-aplicacoes}

Na gestão de projetos de \textit{software}, é fundamental estabelecer papéis e atribuições claras para cada membro da equipe, pois isso promove a clareza quanto às responsabilidades de cada um e evita conflitos e duplicação de esforços. Além disso, o uso de ferramentas de gestão adequadas, como sistemas de controle de versão, plataformas de gerenciamento de tarefas e cronogramas, ajuda a manter o projeto organizado e dentro do prazo. Outro aspecto crucial é estabelecer canais de comunicação confiáveis entre os membros da equipe, garantindo que as informações fluam de maneira rápida e precisa, facilitando a colaboração e a resolução de problemas. 


\section{Metodologia de Gestão de Projeto}

A metodologia de gestão de projeto definida pela equipe foi a metodologia \textbf{Scrum}, que proporciona um \textit{framework} flexível e adaptativo, bem alinhado às dinâmicas do desenvolvimento de \textit{software}. No Scrum, a equipe trabalha de maneira iterativa e incremental, garantindo entregas frequentes e incrementos funcionais do produto. Esse processo possibilita a rápida adaptação a mudanças nos requisitos, melhorando a transparência, colaboração e auto-organização entre os membros do time.

Dentro dessa estrutura, foram empregados todos os artefatos principais do Scrum, como o \textbf{Product Backlog}, o \textbf{Sprint Backlog} e o \textbf{Incremento}. Cada um desses artefatos foi implementado com o objetivo de maximizar a visibilidade do progresso e alinhar as expectativas da equipe, conforme descrito a seguir:

\textbf{Product Backlog}: Consistiu na lista de funcionalidades e melhorias desejadas para o projeto. Esse backlog foi continuamente atualizado pela Product Owner, que também priorizou os itens com base nas necessidades dos usuários e nas mudanças no escopo.

\textbf{Sprint Backlog}: Para cada Sprint, foi selecionado um subconjunto de itens do Product Backlog, formando o Sprint Backlog. Esse backlog da Sprint foi ajustado conforme o feedback e refletiu o escopo específico de cada Sprint, servindo como um plano de trabalho que orientava o time no curto prazo.

\textbf{Incremento}: Ao final de cada Sprint, nossa equipe entregou um incremento funcional do produto, testado e pronto para uso. 

Apesar de todos os artefatos do Scrum terem sido implementados, foi necessário ajustar a realização das reuniões diárias devido à disponibilidade de tempo dos membros do time. Em vez de encontros presenciais ou por videoconferência, a equipe optou por realizar as Daily Scrums via mensagens, facilitando a comunicação assíncrona. Essa adaptação foi necessária para acomodar a agenda dos membros e manter o fluxo de informações, permitindo que cada um atualizasse o restante da equipe sobre o progresso e desafios diários.

O cronograma do projeto completo referente ao segundo semestre de 2024, está presente no Apêndice \ref{apendice_i}.


 A coordenação das tarefas apresentadas no apêndice  \ref{apendice_i} evidencia uma gestão bem estruturada, onde as atividades são planejadas de maneira a evitar sobreposições excessivas e gargalos. Essa organização não só melhora a eficiência da equipe, mas também minimiza o risco de atrasos, assegurando que o projeto progrida de forma fluida e dentro dos prazos estipulados.
\newpage

\section{Papéis e Atribuições}

Dentro da metodologia Scrum, existem três papéis principais: o Product Owner, responsável por representar os interesses do cliente e definir as funcionalidades do produto; o Scrum Master, encarregado de garantir que a equipe siga os princípios e práticas do Scrum; e a equipe de desenvolvimento, responsável por realizar o trabalho necessário para entregar as funcionalidades do produto. Cada um desses papéis desempenha uma função específica no processo de desenvolvimento ágil, contribuindo para a eficiência e sucesso do projeto.

\begin{itemize}
    \item \textbf{Scrum Master:}
O Scrum Master, como líder servo da equipe, desempenha um papel crucial na remoção de obstáculos que possam impedir o progresso do projeto. Sua responsabilidade inclui facilitar reuniões, garantir a adesão aos princípios e práticas do Scrum, e promover um ambiente de trabalho colaborativo e auto-organizável. Além disso, o Scrum Master atua como um mentor, capacitando a equipe a tomar decisões e resolver problemas de forma independente. 
    \begin{itemize}
        \item Dentro do nosso projeto, esse papel foi desempenhado pela integrante Tamiris Delfino.
    \end{itemize}
    
    \item \textbf{Product Owner:}
O Product Owner é o representante dos \textit{stakeholders} e tem a responsabilidade de maximizar o valor do produto. Isso envolve a definição e priorização do \textit{backlog} do produto, garantindo que os requisitos do cliente sejam compreendidos e atendidos pela equipe de desenvolvimento. O Product Owner deve ter uma visão clara dos objetivos do negócio e trabalhar em estreita colaboração com a equipe para garantir que o produto seja desenvolvido de acordo com esses objetivos.
\begin{itemize}
        \item Dentro do nosso projeto, esse papel foi desempenhado pela integrante Letícia Baião.
    \end{itemize}

    \item \textbf{Desenvolvedores:} 
Os Desenvolvedores são os membros da equipe responsáveis por transformar os itens do \textit{backlog} do produto em incrementos potencialmente entregáveis do produto. Eles são auto-organizáveis e colaboram de perto com o Product Owner para entender os requisitos do produto. Os Desenvolvedores têm a responsabilidade de garantir que o trabalho seja concluído dentro do prazo estabelecido para cada Sprint, bem como de buscar continuamente maneiras de melhorar o processo de desenvolvimento.
    \begin{itemize}
        \item Dentro do nosso projeto, esse papel foi desempenhado por todas as integrantes, dada a exigência de todas se envolverem com o desenvolvimento e o prazo para a entrega do projeto, que torna inviável a codificação ser feita por apenas três integrantes.
    \end{itemize}
\end{itemize}

\subsection{Ferramentas de gestão utilizadas}

No projeto, optou-se por utilizar o \textbf{Jira} como nossa ferramenta de gerenciamento de projeto. O Jira foi escolhido devido à sua ampla gama de funcionalidades que atendem às necessidades específicas de gestão de projetos ágeis. Algumas das características chave do Jira incluem sua capacidade de criar e acompanhar histórias de usuário, gerenciar e priorizar o \textit{backlog} do produto, atribuir tarefas aos membros da equipe, monitorar o progresso do projeto através de quadros Kanban e Scrum, e gerar relatórios sobre o desempenho da equipe e do projeto.

\subsection{Canais de comunicação}

No projeto, o \textbf{WhatsApp} foi utilizado como meio de comunicação para coordenar e agendar nossas reuniões. Já para a realização das reuniões semanais, utilizou-se a plataforma \textbf{Meet}. Através dela, foi possível conduzir nossas reuniões de forma remota, possibilitando a participação de todos os membros da equipe, mesmo em localidades diferentes.

Para facilitar o compartilhamento e acompanhamento do histórico de desenvolvimento do projeto, foi criado o blog ADA'S TECH IFSP. Este blog serve como um repositório de informações sobre a evolução do projeto, permitindo o acompanhamento das etapas. O acesso ao blog está disponível na Figura \ref{fig:qrcodeBlog} e no link.



\begin{figure}[ht]
        \centering
        \href{https://adastech.wordpress.com/}{
        \includegraphics[scale=0.5]{images/qrCode-blog.png}
        }        
        \caption{QRcode URL do Blog}
        \label{fig:qrcodeBlog}
    \end{figure}
\href{https://adastech.wordpress.com/}{https://adastech.wordpress.com/}

Através do QRcode da \ac{url} do blog, é possível acessar um dos canais de comunicação do grupo Ada's Tech com o público geral. 

\newpage
Além disso, foi criado o canal no YouTube @AdasTechIFSP, destinado à publicação de todos os vídeos produzidos ao longo do desenvolvimento do projeto, incluindo apresentações e vídeos do Gource. O acesso ao canal está disponível na Figura \ref{fig:qrCodeYoutube}

\begin{figure}[ht]
        \centering
        \href{https://www.youtube.com/@AdasTechIFSP}{
        \includegraphics[scale=0.5]{images/qrCode-yotube.png}
        }        
        \caption{QRcode URL do Canal no YouTube}
        \label{fig:qrCodeYoutube}
    \end{figure}
    \href{https://www.youtube.com/@AdasTechIFSP}{https://www.youtube.com/@AdasTechIFSP}


    Esses canais de comunicação garantem a eficiência na troca de informações e documentam o progresso do projeto de forma acessível e organizada. O blog e o canal no YouTube são ferramentas que mantêm todos os membros e interessados informados sobre as atividades e avanços do projeto.