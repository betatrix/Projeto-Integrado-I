
\newglossaryentry{analise-estatica}{
                name={análise estática},
                description={ferramenta que analisa o código sem executar de forma a detectar possíveis problemas
                } }

\newglossaryentry{api}{
             name={API},
             description={Application Programming Interface (Interface de programação de Aplicação) é uma interface de programação que permite que diferentes módulos de um sistema (ou sistemas diferentes) se comuniquem utilizando um padrão pré definido
                 } }

\newglossaryentry{backup}{
                name={\textit{backup}},
                plural={\textit{backups}},
                description={cópia de segurança de dados
                } }

\newglossaryentry{commit}{
                name={\textit{commit}},
                plural={\textit{commits}},
                description={operação de efetivação de uma alteração, em um banco de dados efetiva uma transação, em um repositório de controle de versão grava um conjunto de alterações que foram efetuadas
                } }

\newglossaryentry{gource}{
                name={gource},
                description={ferramenta que permite gerar um vídeo a partir dos logs de um repositório de controle de versão
                } }

\newglossaryentry{integracao-continua}{
                name={integração contínua},
                description={ferramenta que permite integrar o ambiente de desenvolvimento desde o \gls{vc} até a implantação nos servidores de teste/produção executando todos os processos definidos  incluindo \gls{analise-estatica}, \gls{testes-automatizados} e outros
                } }

\newglossaryentry{latex}{
                name={\LaTeX},
                description={ferramenta que permite gerar documentos sem grandes preocupações com formatação, permite definir funções, macros e outros elementos como um programa
                } }

\newglossaryentry{latexdiff}{
                name={latexdiff},
                description={ferramenta que permite gerar um documento indicando as diferenças entre duas versões de um documento \gls{latex}
                } }

\newglossaryentry{openapi}{
                name={OpenAPI},
                description={Padrão para definição de especificações de API, esse padrão foi definido como uma evolução da ferramenta Swagger
                } }
                
\newglossaryentry{overleaf}{
                name={overleaf},
                description={ferramenta online para edição de documentos \gls{latex}, possui algumas limitações para ter acesso a mais recursos sem custo é recomendável utilizar uma ambiente instalado localmente no computador
                } }
                
                
\newglossaryentry{svn}{
                name={subversion},
                description={um sistema de \gls{vc} centralizado} }
                
\newglossaryentry{testes}{
                name={testes},
                description={procedimentos executados no software de forma a validar se o software está executando de acordo com as definições e requisitos
                } }
                
\newglossaryentry{testes-automatizados}{
                name={testes automatizados},
                description={são \gls{testes} executados de forma automatizada sem intervenção humana, são programados durante o desenvolvimento e executados sempre que o software for alterado de forma a garantir que o software continue executado da forma esperada
                } }
                
\newglossaryentry{vc}{
                name={controle de versão},
                description={sistema que permite controlar versões de arquivos, normalmente utilizado para versionamento de software, mas qualquer arquivo pode ser controlado. Em arquivos do tipo texto é possível fazer comparação entre versões diferentes. Existem  sistemas distribuídos e centralizados
                } }

\newglossaryentry{HTML}{
    name={HTML},
    description={Linguagem de Marcação de Hipertexto (HyperText Markup Language), usada para estruturar e apresentar conteúdo na web}
}

\newglossaryentry{SSL}{
    name={SSL},
    description={Protocolo de segurança (Secure Sockets Layer) que garante a comunicação criptografada entre servidores e clientes na internet}
}

\newglossaryentry{HTTP}{
    name={HTTP},
    description={Protocolo de Transferência de Hipertexto (HyperText Transfer Protocol), usado para comunicação entre navegadores e servidores web}
}

\newglossaryentry{E2E}{
    name={E2E},
    description={ end-to-end (E2E)  é um tipo de teste que valida o funcionamento completo de uma aplicação, simulando fluxos reais de uso do inicio ao fim.}
}

