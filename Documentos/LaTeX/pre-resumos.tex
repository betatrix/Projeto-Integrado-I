% ---
% RESUMOS
% ---

% resumo em português
\setlength{\absparsep}{18pt} % ajusta o espaçamento dos parágrafos do resumo
\begin{resumo}
\todo{fazer o seu resumo}
 Segundo a \citeonline[3.1-3.2]{NBR6028:2003}\index{ABNT}\index{NBR6028}, o resumo\index{resumo} deve ressaltar o objetivo, o método, os resultados e as conclusões do documento. A ordem e a extensão destes itens dependem do tipo de resumo (informativo ou indicativo) e do  tratamento que cada item recebe no documento original. 
O resumo deve ter um paragrafo único e deve ter entre 150 e 500 palavras para trabalhos acadêmicos ou entre 100 e 250 para artigos de periódicos. 
O resumo deve ser precedido da referência do documento, com exceção do resumo inserido no  próprio documento. (\ldots) As palavras-chave devem figurar logo abaixo do resumo, antecedidas da expressão Palavras-chave:, separadas entre si por ponto e finalizadas também por ponto.

 \textbf{Palavras-chaves}: latex. abntex. editoração de texto.
\end{resumo}

% resumo em inglês
\begin{resumo}[Abstract]
 \begin{otherlanguage*}{english}
   This is the english abstract.
\todo{fazer tradução do resumo, não utilizar tradução automática}
   \vspace{\onelineskip}
 
   \noindent 
   \textbf{Key-words}: latex. abntex. text editoration.
 \end{otherlanguage*}
\end{resumo}