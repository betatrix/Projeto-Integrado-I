% \section{Papéis e responsabilidades}

% Na disciplina existem diversos papéis que são assumidos pelos participantes, o entendimento desses papéis permite atingir corretamente os objetivos da disciplina:
% \begin{itemize}
%     \item \textbf{Estudante} - Deve desenvolver as atividades da disciplina seguindo os preceitos deste documento e orientações passadas pelos professores, colaborando para o sucesso do projeto desenvolvido pela equipe;


%     \item \textbf{Equipe} - Segundo \cite{EQUIPES}: 
%     \begin{citacao}“Um grupo de pessoas com alto grau de interdependência está direcionado para a realização de uma meta ou para a conclusão de uma tarefa, cria-se o conceito de EQUIPE. Em outra palavras, membros de uma equipe concordam com uma meta e concordam que a única maneira de alcançar essa meta é trabalhar em conjunto". 
%     \end{citacao}
%     Desta forma, as equipes são compostas por um número definido de estudantes, que tem como objetivo a concretização do trabalho da disciplina.
    
%     Algumas outras definições de equipes e vídeos de apoio podem ser encontrados em: \dicasIvan{equipes}
    

%     \item \textbf{Professor} - Tem o papel de orientar e avaliar, buscando atingir os objetivos da disciplina;
    
%     \justificativa{Eles precisam buscar informações antes de nos questionar e vamos atender de acordo com requisições, não devemos influenciar nas decisões se não formos questionados ou consultados}
    

%     \item \textbf{Cliente} - Os professores assumem o papel de cliente do projeto e devem ser consultados como um cliente real. Ao desenvolver uma aplicação para resolver um problema real e tendo acesso a usuários reais o projeto pode evoluir muito pois passa por diversas visões em relação ao problema. Uma equipe também pode assumir o papel de cliente de outra equipe desde que não entrem em conflito com as decisões dos clientes principais que são os Professores;

%     \item \textbf{Banca Examinadora} - O trabalho é apresentado para uma banca que vai avaliar tanto os documentos demonstrando o desenvolvimento como a aplicação em execução. Essa banca é composta pelos professores da disciplina e convidado(s).
    
% \end{itemize}

