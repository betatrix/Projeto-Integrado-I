\section{Comunicação entre alunos e professores}
Existem diversos canais de comunicação que poderão ser utilizados durante o desenvolvimento da disciplina, conforme a seguir descrito.
\begin{itemize}
    \item Comunicador do \ac{suap} onde os professores muitas vezes enviam comunicados oficiais que precisam de registro;
    
    \item Curso definido no Moodle da disciplina onde algumas atividades devem ser entregues e também permite o envio de mensagens;
    
    \item E-mail dos alunos para os professores com dúvidas especificas (sempre enviar com cópia para ambos professores e indicar claramente qual a turma/equipe que faz parte, pois os professores tem projetos de diversas turmas e nem sempre os e-mails chegam com o nome correto do aluno);
    
    \item Grupo no Telegram que permite a comunicação entre todos participantes da turma, dúvidas genéricas devem ser feitas principalmente por esse canal pois permitem que todos tenham acesso a informação. Nesse grupo são enviados os links para as aulas/conversas síncronas da disciplina;
    
    \item Ferramentas de conferencia (Meet, RNP, Teams, Telegram  etc) para conversas síncronas.
    
\end{itemize}

É importante lembrar que algumas ferramentas devem ser utilizadas com cuidado, não se deve enviar mensagem com notificação de madrugada por exemplo. No Telegram é possível agendar o envio de uma mensagem ou até enviar sem a notificação, bastando escolher isso no momento do envio.

Existe um canal genérico de projetos (IFSP-SPO-Projetos) onde algumas informações gerais que atendem alunos do ensino médio e superior são publicadas \url{https://t.me/joinchat/AAAAAET-oEt6v2nyhgx2CQ}.





\justificativa{O Telegram não dá acesso ao número de telefone se o usuário não desejar, assim respeita um pouco o que a própria escola faz onde não temos acesso aos números, os grupos tem o histórico disponível e assim os alunos tem acesso ao que já aconteceu}


