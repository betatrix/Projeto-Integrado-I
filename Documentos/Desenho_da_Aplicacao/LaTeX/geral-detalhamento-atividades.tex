% \chapter{Informações sobre as atividades desenvolvidas}

% Esse capitulo apresenta as atividades a serem desenvolvidas durante o projeto, as seções em sua maioria seguem a ordem de desenvolvimento. Mas algumas atividades podem acontecer paralelamente a outras, e em alguns casos a preparação para o desenvolvimento da atividade deve ser feito antecipadamente.

% \section{Definição da equipe}\label{atv-definicao-equipe}
% Os alunos devem se organizar em equipes de acordo com o número definido pelos professores no primeiro dia de aula. Normalmente as equipes variam entre 4 e 6 participantes de acordo com o número de alunos em cada turma. 

% \todo[inline]{Aqui a definição do gerente e escolha das equipes seria o ideal, mas nunca aceitam fazer, se tiverem alguma sugestão é interessante colocar}

% \todo[inline]{Muitos casos escolhem os amigos e quando um realmente assume o papel de gerente depois deixam de ser amigos.... Quando não estão preocupados vão levando de qualquer forma}

% \todo[inline]{a escolha do perfil dos participantes é importante, mas normalmente só escolhem os amigos}



% \section{Planejamento do projeto}\label{atv-planejamento-projeto}

% Para se obter o resultado desejado e chegar ao objetivo é importante um planejamento adequado e um acompanhamento da evolução desse plano para chegar ao objetivo. Algumas  atividades devem ser executadas de forma sequencial e portanto o planejamento correto evita atrasos pois não adianta colocar mais energia/pessoas para reduzir o tempo de determinadas atividades ou executar paralelamente \cite{brooks1995mythical}.
% % Nine people can’t make a baby in a month. (regarding the addition of more programmers to get a project completed faster)



% \section{Escolha do(a) gerente da equipe}\label{atv-escolha-gerente}
% Cada equipe deve definir um Gerente que deve ser responsável por organizar durante todo o período de desenvolvimento do projeto. O Gerente também é responsável por participar das reuniões de gerencia que podem acontecer durante a disciplina, onde os professores podem passar informações para as equipes.


% \section{Envio da linha correspondente ao acessos.txt}\label{atv-acessos}
% Para a criação do diretório correspondente da equipe no repositório de controle de versão \atividade{controle-de-versao}, cada equipe deve enviar para os professores uma linha de texto simples (por e-mail) no mesmo formato do arquivo \textbf{acessos.txt} disponível na raiz do repositório de controle de versão. Os prontuários utilizados nesse arquivo devem ser definidos em letras minúsculas, e o acesso ao repositório sempre deve ser feito com login em letras minúsculas.

% A maior parte das atividades da disciplina depende do acesso a este repositório, sendo importante que essa atividade seja cumprida no menor tempo para garantir o acesso de todos ao repositório.


% \section{Criação de canal no YouTube}\label{atv-criacao-youtube}
% Deve ser criado um canal no YouTube para publicação de todos os vídeos gerados durante o desenvolvimento e apresentações do projeto.

% \begin{itemize}
%     \item todos os vídeos publicados devem ter na descrição e nas tags \textbf{\acs{ifsp}}, \textbf{\acs{spo}} e \textbf{\codigoDisciplina};
    
%     \item vídeos do \gls{gource} devem ser publicados nesse canal \autoref{atv-video-gource}.
% \end{itemize}

% \section{Criação de blog}\label{atv-criacao-blog}
% Deve ser criado um blog para a equipe com as seguintes características:
% \begin{itemize}
%     \item Deve ter um feed \acs{rss} ou Atom, de forma a permitir que as publicações sejam visualizadas na íntegra em ferramentas de consumo destes formatos;
    
%     \item O modelo utilizado para o blog deve ser tal que permita a visualização completa de cada publicação, sem necessidade de clicar em um \enquote{ver mais} ou um ícone para permitir a visualização.
    
% \end{itemize}

% A partir da criação do blog, toda semana deverá ser feita uma publicação de acordo com o descrito na \autoref{atv-publicacao-blog}

% A lista de blogs de projetos anteriores está disponível em: \dicasIvan{blogs-de-trabalhos}, muitos desses blogs possuem relatos e dicas que podem auxiliar no desenvolvimento do seu projeto.


% \section{Criação e atualização do arquivo equipe.yaml}\label{atv-equipe-yaml}
% Deve ser criado um arquivo no diretório da equipe, no repositório de controle de versão, com as informações sobre a equipe e o projeto que vai ser desenvolvido. Esse arquivo deve ser criado a partir da definição da equipe (\autoref{atv-definicao-equipe}) e deverá ser atualizado se houverem mudanças nas informações (ex a primeira versão do arquivo não possui detalhes sobre o projeto a ser desenvolvido, quando o projeto for aprovado pelos professores então o arquivo deverá ser atualizado).

% O arquivo deve seguir o mesmo formato dos arquivos já existentes nos diretórios de equipes de projetos anteriores e ser validado com a ferramenta yamllint \url{https://github.com/adrienverge/yamllint} que é utilizada no servidor. Outras ferramentas que possuem o mesmo nome podem ter critérios diferentes de validação então é importante a utilização da ferramenta sugerida.
% \justificativa{Muitos alunos tentam utilizar um site com esse nome que não valida da mesma forma e o arquivo acaba sendo recusado.  }

% Esse arquivo é a base para definição da página \dicasIvan{blogs-de-trabalhos}, caso a sua equipe não apareça nessa listagem provavelmente o seu arquivo é invalido, e na página existe um link para ver o arquivo combinado que contem as mensagens de erro de validação.

% Esses arquivos são processados diariamente para atualização no site.






% \section{Documentos do projeto}\label{documentos-latex}

% Todos os documentos que serão entregues para avaliação devem ser feitos utilizando o \LaTeX \space de acordo com os padrões da \ac{abnt} e do Guia de Normatização do \ac{ifsp}. Um modelo de documento está disponível em \urlmodelo \space e contem exemplos de diversos itens que normalmente são utilizados nesses documentos. O documento modelo também possui informações sobre procedimentos para revisão dos textos.


% No repositório de controle de versão existem documentos de outras equipes que também podem ser utilizados como referência. O documento de modelo também possui informações sobre o desenvolvimento de software e problemas encontrados em projetos anteriores tanto em documentação como em relação ao desenvolvimento. 

% Na página \dicasIvan{textos} existem diversas informações uteis para facilitar a elaboração do seu documento. Cuidado com plágio \dicasIvan{plagio}, não faça cópias de outros documentos, escreva com suas palavras e referencie a informação.

% A ferramenta overleaf onde o modelo de documento se encontra é uma ferramenta de desenvolvimento de textos \LaTeX, que possui algumas limitações na versão gratuita, é importante que os alunos mantenham um ambiente de compilação em seus computadores de forma a não ficarem dependentes dessa ferramenta \emph{online}. Algumas atividades como a criação do documento de diferenças entre versões \atividade{latexdiff} dependem de um ambiente corretamente configurado e que não funciona diretamente no overleaf. Alguns professores já possuem contas com características adicionais da conta gratuita do overleaf e podem compartilhar projetos com mais recursos, mas isso não deve ser considerado como a solução normal de uso da equipe.
% \justificativa{Aqui é importante que aqueles que não tem os recursos ainda compartilhem os seus links de referencia do overleaf para ganharem esse tipo de acesso, acho que são somente 10 referencias para ganhar histórico e compartilhamento ilimitado}


% \section{Utilização do repositório de controle de versão}
% \label{atv-controle-de-versao}
% \todo{Aqui precisa de ajuste para as disciplinas do superior que podem utilizar repositório externo e precisam sincronizar a cada 15 dias}
% É obrigatório o uso do repositório de controle de versão indicado pelos professores. A \ac{url} para acesso é disponibilizada no moodle da disciplina. O acesso deve ser feito utilizando como login o prontuário do aluno (spXXXXXX) sempre em letras minusculas. A senha é a mesma que é utilizada para acesso na rede sem fio do Campus e pode ser redefinida no site \url{https://central.spo.ifsp.edu.br/}. O repositório é um sistema de controle de versão centralizado Subversion (informações adicionais em \dicasIvan{subversion}).

% O uso correto do repositório permite o desenvolvimento pelos diversos membros da equipe de forma remota mantendo o conteúdo em local de fácil acesso para todos. Os \glspl{commit} devem ser frequentes para garantir que todos tenham acesso ao desenvolvimento atual da equipe, devem seguir as regras indicadas no arquivo \textbf{txt} na raiz do repositório. Cada equipe também deve manter seu próprio \gls{backup} dos dados do repositório.

% Cada \gls{commit} deve ter um comentário descrevendo que que foi alterado, essa informação é muito importante também para criação posterior do vídeo com a ferramenta \gls{gource} \atividade{video-gource}.

% O Código e documentação de projetos anteriores do curso técnico e superior fica disponível para consulta por todos os alunos do Campus São Paulo do \ac{ifsp}.

% A cada entrega/apresentação deve ser feita uma \emph{tag} no repositório indicando a data e tipo de entrega (ex. 2020-02-10-Prova-de-Conceito)\footnote{ \emph{tag} é uma função disponível em sistemas de controle de versão, não significa criar um diretório com o nome indicado}.

% Se o projeto a ser desenvolvido for um desenvolvimento de funções adicionais para projeto de código aberto, será permitida a utilização de um repositório externo público após a avaliação e aprovação dos professores. O repositório oficial do \ac{ifsp} deverá ser ser sincronizado pelo menos uma vez a cada 15 dias com a versão atualizada do repositório externo.

% \justificativa{No técnico que ainda não tem experiencia continuamos com o repositório centralizado}





% \section{Escolha da metodologia de gerenciamento}\label{atv-metodologia-gerenciamento}

% Cada equipe tem liberdade para escolha das metodologias utilizadas durante o projeto, para o controle do projeto deve ser escolhida uma (ou combinação de metodologias) de forma a gerenciar a evolução das atividades. É importante que a equipe realmente siga a metodologia escolhida e utilize os artefatos da metodologia.

% \section{Escolha da metodologia de desenvolvimento}\label{atv-metodologia-desenvolvimento}

% Cada equipe tem liberdade para escolha das metodologias utilizadas durante o desenvolvimento, para o desenvolvimento do software deve ser escolhida uma (ou combinação de metodologias) de forma executar o desenvolvimento. É importante que a equipe realmente siga a metodologia escolhida e utilize os artefatos da metodologia.

% \section{Estudo de tecnologias}\label{atv-estudo-tecnologias}

% Antes de iniciar o projeto deve ser feito um estudo das tecnologias que podem ser aplicadas no projeto, pequenos programas de teste podem ser desenvolvidos nesse momento para verificar a aderência dessa tecnologia aos objetivos do projeto. Tecnologias já conhecidas reduzem os estudos, mas deve-se tomar cuidado pois existem casos onde mesmo com um novo aprendizado outras tecnologias podem ser mais eficientes para o projeto.

% É importante considerar também os custos de determinadas tecnologias ou plataformas. Se o objetivo do projeto é criar uma solução para uma entidade sem fins lucrativos e que não tenha verbas para manter um sistema no ar, não vai adiantar escolher uma plataforma que tenha um custo mensal.
% Além disso algumas plataformas tecnologias possuem limites que podem não ser suficientes para o desenvolvimento da aplicação.

% O \ac{ifsp} possui convênios que permitem acesso a pacotes de serviços de forma gratuita e também existem programas gratuitos dos grandes provedores de nuvem (ex AWS Educate).
% Analisando projetos anteriores \atividade{avaliacao-projetos-anteriores} é possível obter um grande conjunto de ideias sobre tecnologias possíveis de utilização. 



% \section{Definição de tecnologias}\label{atv-definicao-tecnologias}

% Após o estudo das tecnologias passiveis de aplicação no projeto a equipe deve fazer suas escolhas e desenhar o projeto de acordo com as tecnologias estudadas.
% Em alguns casos os provedores de nuvem permitem escolher onde a aplicação será hospedada, Brasil, Europa, América do Norte etc dependendo do tipo de aplicação a localização pode afetar a latência e a experiencia do usuário. 

% As escolhas devem ser justificadas e compatíveis com os objetivos do projeto.
% \melhorar{complementar mais aqui....}


% \section{Proposta inicial}\label{atv-proposta-inicial}
% Cada equipe deve elaborar uma proposta indicando a aplicação que deseja desenvolver, a proposta deve ser desenvolvida e apresentada para a turma. Os professores tem que aprovar o tema e o escopo do projeto a ser desenvolvido. É importante que as aulas anteriores sejam utilizadas para apresentar e negociar com os professores as possíveis propostas de forma que, no dia da apresentação, já seja apresentada a versão final acordada com os os professores.


% \begin{itemize}

%     \item A ordem de apresentação é definida por sorteio;
    
%     \item Documento \ac{pdf} gerado no \LaTeX, com aproximadamente 5 páginas de texto contendo contextualização, ideia, objetivos, análise de concorrentes, referências utilizadas a partir da análise de projetos anteriores \atividade{avaliacao-projetos-anteriores} e indicação de tecnologias que pretendem utilizar \atividade{estudo-tecnologias};
    
%     \item \emph{Slides} para apresentação da proposta inicial \atividade{slides};
    
%     \item Uma apresentação de 10 minutos devera ser feita e será seguida de perguntas/comentários dos professores e alunos da turma;

%     \item Vídeo de 5 minutos no YouTube demonstrando essa proposta \atividade{criacao-youtube};
    
%     \item Auto avaliação da equipe \atividade{planilha-notas} deve ser entregue no repositório.
% \end{itemize}



% \section{Definição de coding convention}\label{atv-coding-convention}

% A equipe deve definir o padrão de codificação que ira utilizar e seguir durante todo o desenvolvimento de forma que todo o código siga um padrão único. Existem ferramentas de \gls{analise-estatica} que também verificam se o formato do código segue um padrão pré-definido, a utilização dessas ferramentas ajuda a seguir esse requisito de desenvolvimento. 

% \section{Modelagem de dados}\label{atv-modelagem-dados}
% A modelagem de dados é fundamental para o correto funcionamento de um software. Sendo assim, é importante que a equipe efetue a modelagem correta seguindo o método pertinente ao seu projeto (seja pela utilização de um \ac{mer}, seja pela utilização de um diagrama de classes, quando utilizado um \ac{orm}). Ainda, não basta que seja feita a modelagem, mas que se olhe de forma criteriosa para o modelo gerado e que ele seja validado com dados reais pertinentes à solução proposta.

% Uma forma simples de verificar se um determinado modelo de dados atente é fazer exercícios simples de inserção, atualização e exclusão considerando os cenários da aplicação, usando ferramentas simples, como planilhas, e, com base nas informações registradas, tentar \enquote{responder}  a consultas que serão relevantes para seus usuários (observando os requisitos do projeto), de forma a ver se a informação correta será obtida como resposta (e fazendo os devidos ajustes no modelo caso contrário).


% \melhorar{falar sobre dados históricos / temporais, eles esquecem coisas básicas como preço que deve ser registrado em uma compra para no futuro saber o valor}

% \section{Codificação da aplicação}\label{atv-codificacao-aplicacao}

% Durante o desenvolvimento da aplicação diversos critérios devem ser seguidos para garantir que o seu código seja claro tanto para os programadores como para os computadores, os computadores entendem qualquer código que siga os critérios de sintaxe definidos da linguagem, mas as pessoas podem ter dificuldade, a utilização de um mesmo padrão por toda a equipe facilita \atividade{coding-convention}, além da utilização de padrões já existentes de estruturas de dados e de codificação. Isso facilita tanto a manutenção por outras pessoas como também pelo próprio desenvolvedor.

% Tenham em mente as seguintes frases\footnote{\url{https://www.javacodegeeks.com/2012/11/20-kick-ass-programming-quotes.html}}:
% \begin{itemize}

% % https://groups.google.com/g/comp.lang.c++/c/rYCO5yn4lXw/m/oITtSkZOtoUJ?pli=1
%     \item \enquote{Always code as if the guy who ends up maintaining your code will be a violent psychopath who knows where you live.} – Martin Golding;
    
%     \item \enquote{Any fool can write code that a computer can understand. Good programmers write code that humans can understand.} – Martin Fowler;
    
%     \item \enquote{Good code is its own best documentation. As you’re about to add a comment, ask yourself, \textquote{How can I improve the code so that this comment isn’t needed?}} – Steve McConnell.
    
% \end{itemize}

% \todo[inline]{Buscar as referencias originais}
% \todo[inline]{também é um bom local para colocar mais informações do Clean Code}

% \section{Criação de ambiente para aplicação}\label{atv-criacao-ambiente}
% A partir das escolhas feitas pela equipe \atividade{definicao-tecnologias} deve ser criado o ambiente no provedor escolhido para a aplicação. O ambiente para desenvolvimento e apresentação pode ser uma versão reduzida do ambiente necessário para a aplicação (ex em produção a aplicação precisa de Balanceador de Carga mas para desenvolvimento isso não é necessário). Além disso o ambiente pode ser criado em localização diferente da ideal para redução de custos.
% Diferenças entre o ambiente proposto para execução da aplicação e o ambiente utilizado devem ser claramente detalhadas na documentação.
% Considerando que os serviços gratuitos possuem limites é possível a utilização de servidores distribuídos entre contas dos membros da equipe. Além disso é importante tomar o cuidado com serviços que fazem cobrança após o limite gratuito de forma a não terem despesas desnecessárias.



% \section{Disponibilização da aplicação na internet}\label{atv-hostname}
% Para que a aplicação fique disponível na internet deverá ser utilizado um nome de acesso \emph{hostname} em vez de endereço IP. Isso garante que mesmo na mudança do local de hospedagem o endereço de acesso permanece inalterado. Poderá ser utilizado um serviço de \ac{dns} dinâmico (dyndns), endereço oferecido pelo provedor de serviço ou um domínio registrado caso a equipe tenha interesse posterior em manter a aplicação (mas não é necessário o registro de domínio próprio para cumprir os requisitos da disciplina)



% \section{Prova de conceito}\label{atv-prova-de-conceito}
% Deverá ser desenvolvida uma Prova de Conceito (\ac{poc}) que deve demonstrar a aderência das tecnologias escolhidas com a aplicação que deve ser desenvolvida. Essa prova de conceito deve demonstrar a comunicação desde o usuário até a base de dados e utilizar de forma simples as tecnologias escolhidas para demonstrar que elas funcionam para o objetivo desejado.
% \todo[inline]{Alguns não entenderam claramente o que deve entrar na POC, precisamos desenhar melhor....}

% \begin{itemize}
%     \item Deve ficar disponível de forma pública na internet;
    
%     \item Utilizar os ambientes/plataformas escolhidos para implantação \atividade{criacao-ambiente};
    
%     \item Comunicação criptografada \atividade{https};
    
%     \item Demonstrar o funcionamento da internacionalização (duas línguas), utilizando corretamente as funcionalidades da linguagem a partir de arquivos de recursos com as definições textuais de cada linguagem (não deve ser utilizada tradução de página via Google tradutor);
    
%     \item Demonstrar que os dados do usuário são enviados para o servidor e armazenados na base de dados;
    
%     \item Demonstrar que APIs especificas atendem as necessidades desejadas;
    
%     \item Auto avaliação da equipe \atividade{planilha-notas};
    
%     \item Um vídeo de aproximadamente 3 minutos demonstrando a aderência desta prova de conceito com a aplicação final \atividade{criacao-youtube};
    
%     \item Um relatório (máximo de 5 páginas) identificando as tecnologias utilizadas e como elas serão utilizadas no projeto (diagrama de arquitetura);
    
%     \item Vídeo do \gls{gource} \atividade{video-gource};
    
%     \item Uma apresentação de 15 minutos para professores e alunos;

%     \item \emph{Slides} \atividade{slides};

%     \item A ordem de apresentação é definida por sorteio.
    
    
% \end{itemize}

% \section{Apresentação parcial}\label{atv-apresentacao-parcial}
% A apresentação parcial consiste em mostrar a situação atual do projeto para os professores e alunos das outras equipes, é importante a utilização de um roteiro para demonstrar de maneira eficiente a situação em que o desenvolvimento se encontra tanto em termos de aplicação como de documentação.

% \begin{itemize}

%     \item Auto avaliação da equipe \atividade{planilha-notas};
    
%     \item Vídeo do \gls{gource} \atividade{video-gource};
    
%     \item Uma apresentação de 20 minutos para professores e alunos;

%     \item \emph{Slides} \atividade{slides};

%     \item A ordem de apresentação é definida por sorteio.
    
% \end{itemize}





% \section{Vídeo do gource}\label{atv-video-gource}
% O video criado com a representação do desenvolvimento no controle de versão deve ser feito com a ferramenta \gls{gource} e deve considerar os seguintes itens:

% \begin{itemize}
%     \item Utilizar opção \textbf{\emph{--key}};
    
%     \item Utilizar as opções de legendas (\textbf{\emph{caption}}) para registrar as principais mudanças feitas no repositório;
    
%     \item Deve ter aproximadamente 1 minuto para cada bimestre de desenvolvimento;
    
%     \item Alterar os \textbf{\emph{userid}} do repositório pelos nomes dos participantes;
    
%     \item Colocar uma imagem distinta e especifica para cada usuário;
    
%     \item Deve ser publicado no canal do YouTube \atividade{criacao-youtube}.
% \end{itemize}

% A cada entrega de aplicação e no final de cada bimestre deve ser criado um novo vídeo do \gls{gource}.



% \section{Publicação semanal no blog}\label{atv-publicacao-blog}

% O blog tem por objetivo servir como registro das atividades realizadas pela equipe a ser acompanhado pelos docentes. Também é importante como memória da equipe, permitindo a consulta pelos membros da equipe de forma a verificar o que ocorreu em uma determinada semana. Dessa forma, é importante considerar o relato do que realmente ocorreu (inclusive pode-se utilizar a publicação combinada a artefatos da metodologia de gerenciamento de projetos utilizada).

% A cada semana deve ser publicado no blog pelo menos uma vez indicando as atividades desenvolvidas por cada elemento da equipe. Deve indicar claramente quem foi a pessoa que fez a publicação.

% As publicações semanais no blog devem conter um relatório gerencial de andamento do projeto, com identificação do desempenho e das tarefas individuais (ex. reuniões de \emph{milestones}, \emph{project checkpoint}, \emph{sprint review} etc.).


% \section{Comunicação criptografada utilizando \acs{https}/\acs{tls}}\label{atv-https}
% A comunicação entre os  módulos da aplicação  e usuário devem sempre ser feitas de forma segura. Para isso deve ser utilizado o protocolo \ac{https} ou \ac{tls} que permitem a comunicação de forma criptografada. A configuração pode ser feita utilizando um certificado gratuito como o disponibilizado pelo serviço Let's Encrypt ou também utilizando serviços gratuitos como CloudFare, CloudFront etc.

% A configuração desse serviço deve ser avaliada e obter uma nota mínima \textbf{A} \autoref{atv-ssllabs}

% \section{Avaliação da configuração \acs{https}/\acs{tls}}\label{atv-ssllabs}

% É importante garantir a confidencialidade entre o usuário e a aplicação no servidor, para isso deve ser utilizado um protocolo de criptografia, o mais utilizado é o \ac{https} que pode ser configurado com um certificado gratuito como Let's Encrypt, AWS ou similar ou através de um serviço que já ofereça esse acesso criptografado (CloudFront, Cloudflare etc). O processo de criptografia deverá ser documentado e a configuração deverá ser analisada utilizando a ferramenta \url{https://www.ssllabs.com/ssltest/} onde deve obter no mínimo a nota A (existem poucos casos onde pode aceita uma nota inferior, mas esses casos devem ser negociados com os professores).




% \section{Avaliação de respostas HTTP}\label{atv-security-headers}
% Para maior segurança da aplicação e dos usuários existem diversos cabeçalhos que podem ser configurados nas respostas do protocolo \ac{http}, para validação desses cabeçalhos deverá ser utilizada a ferramenta de análise disponível em \url{https://securityheaders.io/}, dependendo do tipo de aplicação as condições podem variar e portanto deverá ter a melhor nota possível nessa ferramenta justificando os resultados na documentação.

% \section{Análise estática}\label{atv-analise-estatica}
% O processo de \gls{analise-estatica} permite avaliar o código fonte da aplicação sem necessidade de execução dos programas por meio de testes \atividade{testes}. É uma atividade incremental que deve ser executada de forma constante de forma a evitar problemas no código. Desenvolver o código de forma limpa e sem erros conhecidos reduz o retrabalho e facilita para todos os membros da equipe.

% \section{Documentação via \gls{openapi}}\label{atv-openapi}

% A definição dos pontos de chamadas da aplicação deve ser feita utilizando o padrão \gls{openapi} (padrão que foi definido a partir da ferramenta Swagger). Com a integração de diversas aplicações via protocolo \ac{http} foi criado um padrão \gls{openapi} que permite a definição de forma padronizada de como os "serviços" da aplicação podem ser executados. Isso facilita também a troca de interfaces da aplicação de uma maneira muito simples e também os testes da aplicação. Apesar de existirem ferramentas especificas para chamadas \ac{http} a definição utilizando um padrão permite a migração entre essas ferramentas de forma transparente. 

% A documentação deve ser publicada juntamente com a aplicação e somente os pontos principais indicados no documento.


% \section{Sistema de log}\label{atv-log}
% A aplicação deverá ser desenvolvida utilizando um ou mais sistema de Log de forma a organizar o registro da aplicação e permitir a configuração do armazenamento de acordo com a plataforma de implantação. A utilização de uma biblioteca já existente facilita no código e também para o administrador. Existem diversas bibliotecas que podem ser utilizadas como: commons-logging, log4j, slf4j, log4php, log4net, logcat etc.


% \section{Testes}\label{atv-testes}
% Testes garantem que a aplicação atenda os requisitos definidos durante o seu desenvolvimento. É importante que os testes sejam executados repetidamente e de forma padronizada para isso é necessário que seja criado um Plano de testes para execução manual e também a criação de testes automatizados de forma a reduzir a carga de trabalho e evitar erros. Os testes devem considerar tanto os casos de sucesso como de erro, não adianta testar somente as condições principais de uso, pois durante o uso normal dados errados serão colocados pelo usuário e essas condições deverão ser verificadas também.





% \section{Testes automatizados}\label{atv-testes-automatizados}

% Os testes automatizados evitam o trabalho manual repetitivo e garantem a maior confiabilidade do sistema. Qualquer um pode e deve executar esses testes sempre que uma nova versão for gerada garantindo que a aplicação esteja se comportando da maneira projetada. Esses testes podem iniciar com testes unitários (que verificam um método/função) e ir até testes complexos de integração onde a aplicação é executada em ambiente de execução de forma a garantir que esteja funcionando corretamente. Para isso devem ser utilizadas ferramentas especificas de automatização de testes pois elas já simplificam esse processo.




% \section{Documento com diferenças - latexdiff}\label{atv-latexdiff}
% De forma a demonstrar facilmente as alterações feitas entre duas versões de um documento, deverá ser utilizada a ferramenta \gls{latexdiff} para criação de um documento que demonstra essas diferenças. Cada vez que um documento atualizado for entregue deve ser gerado o \gls{latexdiff} demonstrando as diferenças para a versão anterior. Por isso é muito importante salvar no repositório de controle de versão a revisão utilizada em cada entrega.

% Apesar de existirem algumas versões disponíveis na internet para esse tipo de comparação o documento deve ser gerado localmente em seus computadores, o processo é muito simples mas depende da preparação do ambiente, por isso é importante a preparação antes da data de entrega do documento. O comando recomendado para execução do \gls{latexdiff} pode ser encontrado em \dicasIvan{latexdiff}.




% \section{Planilha de auto avaliação da equipes}\label{atv-planilha-notas}
% Em cada entrega e apresentação cada equipe deverá preencher uma planilha de avaliação onde cada elemento deve dar uma nota, e justificar, para cada elemento da equipe, inclusive a sua própria nota. A planilha de modelo está disponível em \dicasIvan{projetos-e-apresentacoes}. Essa planilha deverá ser salva em um arquivo \ac{pdf} contendo todas as abas originais da planilha e em uma boa formatação para fácil leitura e análise.


% \justificativa{Muitas vezes eles mandam imprimir sem ajustar as páginas e fica tudo quebrado e difícil de ler}





% \section{Entrega da primeira versão do projeto}\label{atv-primeira-versao}

% Seguindo o plano de aulas do \ac{suap} deverá ser entregue no repositório de controle de versão a documentação do projeto.
% \justificativa{Pelo menos uma pois existem casos onde por causa dos feriados e sincronização entre as turmas acabamos tendo que colocar mais de uma semana. 2021 estamos fazendo isso}

% \begin{itemize}
%     \item Documento \ac{pdf} gerado no \LaTeX, com relatório do desenvolvimento \atividade{relatorio-do-desenvolvimento};
% \end{itemize}

% \section{Apresentação da primeira versão do projeto}\label{atv-apresentacao-primeira-versao}

% Cada equipe deve apresentar para a banca examinadora e alunos da turma. Todas equipes devem observar e analisar as apresentações das outras equipes pois devem fazer um relatório avaliando cada outra equipe (Documentos, Apresentação e Aplicação) \atividade{relatorio-analise-projetos}

% É importante lembrar que o(s) convidado(s) não participaram do processo vão avaliar o trabalho e portanto sem conhecimento prévio do que é a aplicação ou do que foi o desenvolvimento.

% \begin{itemize}

%     \item Slides para apresentação  \atividade{slides};
    
%     \item A apresentação deve ter duração de no máximo 45 minutos, seguida de perguntas dos professores;
    
%     \item Pelo menos 15 minutos do tempo da apresentação deve ser reservado para demonstração da aplicação desenvolvida \atividade{demonstracao-aplicacao};
    
%     \item Vídeo de 10 minutos no YouTube demonstrando a aplicação desenvolvida \atividade{criacao-youtube};
    
%     \item Vídeo do \gls{gource} \atividade{video-gource};
    
%     \item Auto avaliação da equipe \atividade{planilha-notas}.
% \end{itemize}


% \section{Relatório do desenvolvimento}\label{atv-relatorio-do-desenvolvimento}

% O relatório deve seguir o padrão \ac{abnt} no formato disponibilizado, deve conter:

% \begin{itemize}
%     \item Introdução;
    
%     \item Problema a ser solucionado;
    
%     \item Justificativa;
    
%     \item Objetivos do projeto;
    
%     \item Revisão da literatura com os assuntos tratados no trabalho:
%         \begin{itemize}
%             \item Não misturar o desenvolvimento feito pela equipe com os estudos feitos;
            
%             \item Não incluir pesquisa sobre temas das disciplinas do curso, somente temas relacionados ao domínio do problema a ser solucionado;
%         \end{itemize}
        
%     \item Histórico das atividades, incluindo informações sobre a gestão do projeto;
    
%     \item Reuniões;
    
%     \item Desenvolvimentos de código, dados;
    
%     \item Levantamentos;

%     \item Escolhas;
    
%     \item Descartes (mudanças de rumo durante o projeto);

%     \item Problemas ocorridos no desenvolvimento / gerenciamento;
    
%     \item Protocolos;
    
%     \item Modelos de dados, classes, índices de banco de dados etc \atividade{modelagem-dados};
    
%     \item Contribuições efetuadas aos projetos de código aberto utilizados;
    
%     \item Tabela com evolução das métricas do projeto \atividade{tabela-metricas};
    
%     \item informações estatísticas do repositório de controle de versão \atividade{estatisticas-repositorio};
    
%     \item Links do projeto:
%     \begin{itemize}
%         \item Controle de versão;
        
%         \item Vídeos;
        
%         \item Blog;
        
%         \item Aplicação web se for publicada na internet;
        
%         \item etc.
%     \end{itemize}
    
%     \item Anexos / Apêndices: 
%     \begin{itemize}
%         \item Todas as publicações semanais do blog \atividade{criacao-blog};
        
%         \item Documento de aprovação \atividade{proposta-inicial}

%         \item Documento da prova de conceito \atividade{prova-de-conceito}

%         \item Cronogramas;
        
%         \item Planos de teste;
        
%         \item Análise de cobertura dos testes;
        
%         \item Relatório análise estática \atividade{analise-estatica};
        
%         \item Manual de usuário \atividade{manual-usuario};
        
%         \item Manual de técnico \atividade{manual-tecnico}.
        
%     \end{itemize}
% \end{itemize}

% Todos os dados qualitativos e quantitativos sobre o projeto (métricas, estatísticas, análises) devem ser efetivamente coerentes com a realidade.

% Elementos como as publicações do blog devem ser apresentados na ordem cronológica de publicação.



% \section{Tabela com evolução das métricas do projeto}\label{atv-tabela-metricas}

% Para demonstrar a evolução do projeto deverá ser criada uma tabela para demonstrar a evolução das métricas do projeto. Essas métricas devem ser registradas pelo menos uma vez por mês. Caso o projeto tenha módulos distintos (servidor, cliente etc) essa tabela deverá ser separada tendo uma tabela para cada módulo e também tabela com totais. A tabela também poderá ser dividida para facilitar a visualização em tabelas com itens de desenvolvimento, gerenciamento, etc.

% Elementos que devem constar:
% \begin{itemize}
%     \item Reuniões;
    
%     \item Publicações de blog;
    
%     \item Requisitos;
    
%     \item Tamanho do projeto;
    
%     \item Arquivos;
    
%     \item Linhas;

%     \item Classes;
    
%     \item Interfaces;
    
%     \item Métodos;
    
%     \item Atributos;
    
%     \item Testes \atividade{testes}
    
%     \item Testes unitários / automatizados \atividade{testes-automatizados}
%     \begin{itemize}
%         \item Classes de testes;
        
%         \item Quantidade de testes;
        
%         \item Percentual de cobertura;
        
%     \end{itemize}
    
%     \item Dados sobre análise estática \atividade{analise-estatica};
    
%     \item \glspl{commit} \atividade{controle-de-versao};
    
%     \item Entidades de banco de dados;
    
%     \item Imagens;
    
%     \item Sons;
    
%     \item Vídeos gerados;
    
%     \item Outros elementos pertinentes para o projeto ou gerenciamento.
% \end{itemize}


% \section{Informações de utilização do repositório}\label{atv-estatisticas-repositorio}

% Para demonstrar a utilização do repositório de controle de versão devem ser apresentadas informações estatísticas geradas pela ferramenta StatSVN. Os dados devem ser comentados. Alguns gráficos são obrigatórios:
%     \begin{itemize}
%         \item Todos commits por autor \waUrl{https://statsvn.org/statsvn/commitscatterauthors.png};
        
%         \item Atividade por autor \waUrl{http://statsvn.org/statsvn/activity.png};
        
%         \item Atividade por dia da semana
%         \waUrl{http://statsvn.org/statsvn/activity\_day.png};
        
%         \item Atividade por hora do dia \waUrl{http://statsvn.org/statsvn/activity\_time.png}.
%     \end{itemize}

% Antes de gerar os gráficos os logs do repositório devem ser ajustados de forma a indicar claramente qual foi a pessoa que fez as atividades e não somente o login das pessoas. Dependendo da versão utilizada de subversion existe um bug que já foi detectado e que pode ser contornado facilmente (informações em \dicasIvan{subversion}).
%     \todo[inline]{Talvez colocar colocar as imagens do statsvn aqui no documento}




% \section{Manual de usuário}\label{atv-manual-usuario}
% Para que o usuário possa utilizar o sistema deverá ser criado um ou mais manuais de usuário (de acordo com possíveis perfis de usuário na aplicação). É bem claro que determinadas aplicações devem ser executadas sem que o usuário tenha que ler um manual, algumas aplicações tem que ter uma interface muito simples para que o usuário não precise ler o manual e para alguns outros casos pequenos vídeos de introdução ou tutoriais dentro da aplicação podem ser suficientes. Existem casos onde é necessário um manual especifico para administrados, moderador etc. A determinação da necessidade do manual escrito ou não deve ser negociada individualmente entre cada equipe e os professores/clientes.


% \section{Manual técnico}\label{atv-manual-tecnico}

% O Manual técnico é aquele que permite que desenvolvedores possam fazer manutenções e atualizações no sistema e o administrador de sistemas possa fazer a implantação, configuração, backup e gerenciamento geral. Esse documento pode fazer parte do relatório geral do desenvolvimento ou ser um documento separado. Caso a equipe opte por um manual separado é importante organizar as informações de forma a não ficar repetindo entre os documentos.



% \section{Demonstração da aplicação desenvolvida}\label{atv-demonstracao-aplicacao}

% Para demonstrar o desenvolvimento a aplicação deve ser apresentada em funcionamento, é importante que seja definido um roteiro para demonstrar as principais funcionalidades e processos da aplicação, a aplicação deve ser populada previamente com dados suficientes para demonstrar o seu funcionamento e entendimento correto, os dados tem que ser compatíveis com o objetivo da aplicação. 

% Durante uma apresentação com tempo limitado pontos básicos de cadastro devem ser limitados a demonstrar alguns elementos e validações, os processos e funções que a aplicação faz para resolver problemas dos usuários devem ser o foco.

% Em alguns casos (como por exemplo utilização de \ac{gps}) alguns recursos não podem ser apresentados em tempo real e portando podem ser apresentados por meio de vídeos gravados com antecedência.






% \section{Slides}\label{atv-slides}
% Para apresentações devem ser gerados \emph{slides} que tem que ser entregues também no repositório de controle de versão. Esses \emph{slides} devem possuir numeração e podem seguir o modelo que está disponível em \urlmodeloApresentacao \space esse modelo foi desenvolvido também com \LaTeX \space de forma que todos os elementos utilizados na documentação podem ser reaproveitados. Apesar dos \emph{slides} serem um roteiro a seguir durante as apresentações é importante que antes da apresentação seja feito um treino para verificar a sequencia e o tempo da apresentação são compatíveis com o solicitado. Todos \emph{slides} que forem utilizados nas apresentações também devem ser colocados em formato \ac{pdf} no repositório.

% A sua apresentação será vista por pessoas que não participaram do desenvolvimento, é importante a contextualização/introdução do que será apresentado. 

